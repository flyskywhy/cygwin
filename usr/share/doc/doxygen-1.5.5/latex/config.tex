\hypertarget{config_config_format}{}\section{Format}\label{config_config_format}
A configuration file is a free-form ASCII text file with a structure that is similar to that of a Makefile, with the default name {\tt Doxyfile}. It is parsed by {\tt doxygen}. The file may contain tabs and newlines for formatting purposes. The statements in the file are case-sensitive. Comments may be placed anywhere within the file (except within quotes). Comments begin with the \# character and end at the end of the line.

The file essentially consists of a list of assignment statements. Each statement consists of a {\tt TAG\_\-NAME} written in capitals, followed by the {\tt =} character and one or more values. If the same tag is assigned more than once, the last assignment overwrites any earlier assignment. For options that take a list as their argument, the {\tt +=} operator can be used instead of {\tt =} to append new values to the list. Values are sequences of non-blanks. If the value should contain one or more blanks it must be surrounded by quotes ("..."). Multiple lines can be concatenated by inserting a backslash ($\backslash$) as the last character of a line. Environment variables can be expanded using the pattern {\tt \$(ENV\_\-VARIABLE\_\-NAME)}.

You can also include part of a configuration file from another configuration file using a {\tt @INCLUDE} tag as follows: 

\footnotesize\begin{verbatim}
@INCLUDE = config_file_name
\end{verbatim}
\normalsize
 The include file is searched in the current working directory. You can also specify a list of directories that should be searched before looking in the current working directory. Do this by putting a {\tt @INCLUDE\_\-PATH} tag with these paths before the {\tt @INCLUDE} tag, e.g: 

\footnotesize\begin{verbatim}
@INCLUDE_PATH = my_config_dir
\end{verbatim}
\normalsize


The configuration options can be divided into several categories. Below is an alphabetical index of the tags that are recognized followed by the descriptions of the tags grouped by category.

\footnotesize
\begin{multicols}{2}
\begin{CompactList}
\item \contentsline{section}{ABBREVIATE\_\-BRIEF}{\ref{config_cfg_abbreviate_brief}}{}
\item \contentsline{section}{ALIASES}{\ref{config_cfg_aliases}}{}
\item \contentsline{section}{ALLEXTERNALS }{\ref{config_cfg_allexternals}}{}
\item \contentsline{section}{ALPHABETICAL\_\-INDEX }{\ref{config_cfg_alphabetical_index}}{}
\item \contentsline{section}{ALWAYS\_\-DETAILED\_\-SEC}{\ref{config_cfg_always_detailed_sec}}{}
\item \contentsline{section}{BINARY\_\-TOC}{\ref{config_cfg_binary_toc}}{}
\item \contentsline{section}{BUILTIN\_\-STL\_\-SUPPORT}{\ref{config_cfg_builtin_stl_support}}{}
\item \contentsline{section}{BRIEF\_\-MEMBER\_\-DESC }{\ref{config_cfg_brief_member_desc}}{}
\item \contentsline{section}{CALL\_\-GRAPH}{\ref{config_cfg_call_graph}}{}
\item \contentsline{section}{CALLER\_\-GRAPH}{\ref{config_cfg_caller_graph}}{}
\item \contentsline{section}{CASE\_\-SENSE\_\-NAMES }{\ref{config_cfg_case_sense_names}}{}
\item \contentsline{section}{CHM\_\-FILE}{\ref{config_cfg_chm_file}}{}
\item \contentsline{section}{CLASS\_\-DIAGRAMS }{\ref{config_cfg_class_diagrams}}{}
\item \contentsline{section}{CLASS\_\-GRAPH}{\ref{config_cfg_class_graph}}{}
\item \contentsline{section}{COLLABORATION\_\-GRAPH}{\ref{config_cfg_collaboration_graph}}{}
\item \contentsline{section}{COLS\_\-IN\_\-ALPHA\_\-INDEX}{\ref{config_cfg_cols_in_alpha_index}}{}
\item \contentsline{section}{COMPACT\_\-LATEX }{\ref{config_cfg_compact_latex}}{}
\item \contentsline{section}{COMPACT\_\-RTF}{\ref{config_cfg_compact_rtf}}{}
\item \contentsline{section}{CPP\_\-CLI\_\-SUPPORT}{\ref{config_cfg_cpp_cli_support}}{}
\item \contentsline{section}{CREATE\_\-SUBDIRS}{\ref{config_cfg_create_subdirs}}{}
\item \contentsline{section}{DETAILS\_\-AT\_\-TOP }{\ref{config_cfg_details_at_top}}{}
\item \contentsline{section}{DIRECTORY\_\-GRAPH}{\ref{config_cfg_directory_graph}}{}
\item \contentsline{section}{DISABLE\_\-INDEX }{\ref{config_cfg_disable_index}}{}
\item \contentsline{section}{DISTRIBUTE\_\-GROUP\_\-DOC}{\ref{config_cfg_distribute_group_doc}}{}
\item \contentsline{section}{DOCSET\_\-BUNDLE\_\-ID}{\ref{config_cfg_docset_bundle_id}}{}
\item \contentsline{section}{DOCSET\_\-FEEDNAME}{\ref{config_cfg_docset_feedname}}{}
\item \contentsline{section}{DOT\_\-GRAPH\_\-MAX\_\-NODES}{\ref{config_cfg_dot_graph_max_nodes}}{}
\item \contentsline{section}{DOT\_\-IMAGE\_\-FORMAT}{\ref{config_cfg_dot_image_format}}{}
\item \contentsline{section}{DOT\_\-MULTI\_\-TARGETS}{\ref{config_cfg_dot_multi_targets}}{}
\item \contentsline{section}{DOT\_\-PATH}{\ref{config_cfg_dot_path}}{}
\item \contentsline{section}{DOT\_\-TRANSPARENT}{\ref{config_cfg_dot_transparent}}{}
\item \contentsline{section}{DOTFILE\_\-DIRS}{\ref{config_cfg_dotfile_dirs}}{}
\item \contentsline{section}{DOXYFILE\_\-ENCODING}{\ref{config_cfg_doxyfile_encoding}}{}
\item \contentsline{section}{ENABLE\_\-PREPROCESSING}{\ref{config_cfg_enable_preprocessing}}{}
\item \contentsline{section}{ENUM\_\-VALUES\_\-PER\_\-LINE}{\ref{config_cfg_enum_values_per_line}}{}
\item \contentsline{section}{ENABLED\_\-SECTIONS}{\ref{config_cfg_enabled_sections}}{}
\item \contentsline{section}{EXAMPLE\_\-PATH}{\ref{config_cfg_example_path}}{}
\item \contentsline{section}{EXAMPLE\_\-PATTERNS}{\ref{config_cfg_example_patterns}}{}
\item \contentsline{section}{EXAMPLE\_\-RECURSIVE}{\ref{config_cfg_example_recursive}}{}
\item \contentsline{section}{EXCLUDE}{\ref{config_cfg_exclude}}{}
\item \contentsline{section}{EXCLUDE\_\-PATTERNS}{\ref{config_cfg_exclude_patterns}}{}
\item \contentsline{section}{EXCLUDE\_\-SYMLINKS}{\ref{config_cfg_exclude_symlinks}}{}
\item \contentsline{section}{EXPAND\_\-AS\_\-DEFINED}{\ref{config_cfg_expand_as_defined}}{}
\item \contentsline{section}{EXPAND\_\-ONLY\_\-PREDEF}{\ref{config_cfg_expand_only_predef}}{}
\item \contentsline{section}{EXTERNAL\_\-GROUPS}{\ref{config_cfg_external_groups}}{}
\item \contentsline{section}{EXTRA\_\-PACKAGES}{\ref{config_cfg_extra_packages}}{}
\item \contentsline{section}{EXTRACT\_\-ALL}{\ref{config_cfg_extract_all}}{}
\item \contentsline{section}{EXTRACT\_\-ANON\_\-NSPACES}{\ref{config_cfg_extract_anon_nspaces}}{}
\item \contentsline{section}{EXTRACT\_\-LOCAL\_\-CLASSES}{\ref{config_cfg_extract_local_classes}}{}
\item \contentsline{section}{EXTRACT\_\-LOCAL\_\-METHODS}{\ref{config_cfg_extract_local_methods}}{}
\item \contentsline{section}{EXTRACT\_\-PRIVATE}{\ref{config_cfg_extract_private}}{}
\item \contentsline{section}{EXTRACT\_\-STATIC}{\ref{config_cfg_extract_static}}{}
\item \contentsline{section}{FILE\_\-PATTERNS}{\ref{config_cfg_file_patterns}}{}
\item \contentsline{section}{FILE\_\-VERSION\_\-FILTER}{\ref{config_cfg_file_version_filter}}{}
\item \contentsline{section}{FILTER\_\-PATTERNS}{\ref{config_cfg_filter_patterns}}{}
\item \contentsline{section}{FILTER\_\-SOURCE\_\-FILES}{\ref{config_cfg_filter_source_files}}{}
\item \contentsline{section}{FULL\_\-PATH\_\-NAMES}{\ref{config_cfg_full_path_names}}{}
\item \contentsline{section}{GENERATE\_\-AUTOGEN\_\-DEF}{\ref{config_cfg_generate_autogen_def}}{}
\item \contentsline{section}{GENERATE\_\-BUGLIST}{\ref{config_cfg_generate_buglist}}{}
\item \contentsline{section}{GENERATE\_\-CHI}{\ref{config_cfg_generate_chi}}{}
\item \contentsline{section}{GENERATE\_\-DEPRECIATEDLIST}{\ref{config_cfg_generate_deprecatedlist}}{}
\item \contentsline{section}{GENERATE\_\-DOCSET}{\ref{config_cfg_generate_docset}}{}
\item \contentsline{section}{GENERATE\_\-HTML}{\ref{config_cfg_generate_html}}{}
\item \contentsline{section}{GENERATE\_\-HTMLHELP}{\ref{config_cfg_generate_htmlhelp}}{}
\item \contentsline{section}{GENERATE\_\-LATEX}{\ref{config_cfg_generate_latex}}{}
\item \contentsline{section}{GENERATE\_\-LEGEND}{\ref{config_cfg_generate_legend}}{}
\item \contentsline{section}{GENERATE\_\-MAN}{\ref{config_cfg_generate_man}}{}
\item \contentsline{section}{GENERATE\_\-PERLMOD}{\ref{config_cfg_generate_perlmod}}{}
\item \contentsline{section}{GENERATE\_\-RTF}{\ref{config_cfg_generate_rtf}}{}
\item \contentsline{section}{GENERATE\_\-TAGFILE}{\ref{config_cfg_generate_tagfile}}{}
\item \contentsline{section}{GENERATE\_\-TESTLIST}{\ref{config_cfg_generate_testlist}}{}
\item \contentsline{section}{GENERATE\_\-TODOLIST}{\ref{config_cfg_generate_todolist}}{}
\item \contentsline{section}{GENERATE\_\-TREEVIEW}{\ref{config_cfg_generate_treeview}}{}
\item \contentsline{section}{GENERATE\_\-XML}{\ref{config_cfg_generate_xml}}{}
\item \contentsline{section}{GRAPHICAL\_\-HIERARCHY}{\ref{config_cfg_graphical_hierarchy}}{}
\item \contentsline{section}{GROUP\_\-GRAPHS}{\ref{config_cfg_group_graphs}}{}
\item \contentsline{section}{HAVE\_\-DOT}{\ref{config_cfg_have_dot}}{}
\item \contentsline{section}{HHC\_\-LOCATION}{\ref{config_cfg_hhc_location}}{}
\item \contentsline{section}{HIDE\_\-FRIEND\_\-COMPOUNDS}{\ref{config_cfg_hide_friend_compounds}}{}
\item \contentsline{section}{HIDE\_\-IN\_\-BODY\_\-DOCS}{\ref{config_cfg_hide_in_body_docs}}{}
\item \contentsline{section}{HIDE\_\-SCOPE\_\-NAMES}{\ref{config_cfg_hide_scope_names}}{}
\item \contentsline{section}{HIDE\_\-UNDOC\_\-CLASSES}{\ref{config_cfg_hide_undoc_classes}}{}
\item \contentsline{section}{HIDE\_\-UNDOC\_\-MEMBERS}{\ref{config_cfg_hide_undoc_members}}{}
\item \contentsline{section}{HIDE\_\-UNDOC\_\-RELATIONS}{\ref{config_cfg_hide_undoc_relations}}{}
\item \contentsline{section}{HTML\_\-ALIGN\_\-MEMBERS}{\ref{config_cfg_html_align_members}}{}
\item \contentsline{section}{HTML\_\-DYNAMIC\_\-SECTIONS}{\ref{config_cfg_html_dynamic_sections}}{}
\item \contentsline{section}{HTML\_\-FOOTER}{\ref{config_cfg_html_footer}}{}
\item \contentsline{section}{HTML\_\-HEADER}{\ref{config_cfg_html_header}}{}
\item \contentsline{section}{HTML\_\-OUTPUT}{\ref{config_cfg_html_output}}{}
\item \contentsline{section}{HTML\_\-STYLESHEET}{\ref{config_cfg_html_stylesheet}}{}
\item \contentsline{section}{IGNORE\_\-PREFIX}{\ref{config_cfg_ignore_prefix}}{}
\item \contentsline{section}{IMAGE\_\-PATH}{\ref{config_cfg_image_path}}{}
\item \contentsline{section}{INCLUDE\_\-GRAPH}{\ref{config_cfg_include_graph}}{}
\item \contentsline{section}{INCLUDE\_\-PATH}{\ref{config_cfg_include_path}}{}
\item \contentsline{section}{INHERIT\_\-DOCS}{\ref{config_cfg_inherit_docs}}{}
\item \contentsline{section}{INLINE\_\-INFO}{\ref{config_cfg_inline_info}}{}
\item \contentsline{section}{INLINE\_\-INHERITED\_\-MEMB}{\ref{config_cfg_inline_inherited_memb}}{}
\item \contentsline{section}{INLINE\_\-SOURCES}{\ref{config_cfg_inline_sources}}{}
\item \contentsline{section}{INPUT}{\ref{config_cfg_input}}{}
\item \contentsline{section}{INPUT\_\-ENCODING}{\ref{config_cfg_input_encoding}}{}
\item \contentsline{section}{INPUT\_\-FILTER}{\ref{config_cfg_input_filter}}{}
\item \contentsline{section}{INTERNAL\_\-DOCS}{\ref{config_cfg_internal_docs}}{}
\item \contentsline{section}{JAVADOC\_\-AUTOBRIEF}{\ref{config_cfg_javadoc_autobrief}}{}
\item \contentsline{section}{LATEX\_\-BATCHMODE}{\ref{config_cfg_latex_batchmode}}{}
\item \contentsline{section}{LATEX\_\-CMD\_\-NAME}{\ref{config_cfg_latex_cmd_name}}{}
\item \contentsline{section}{LATEX\_\-HEADER}{\ref{config_cfg_latex_header}}{}
\item \contentsline{section}{LATEX\_\-HIDE\_\-INDICES}{\ref{config_cfg_latex_hide_indices}}{}
\item \contentsline{section}{LATEX\_\-OUTPUT}{\ref{config_cfg_latex_output}}{}
\item \contentsline{section}{MACRO\_\-EXPANSION}{\ref{config_cfg_macro_expansion}}{}
\item \contentsline{section}{MAKEINDEX\_\-CMD\_\-NAME}{\ref{config_cfg_makeindex_cmd_name}}{}
\item \contentsline{section}{MAN\_\-EXTENSION}{\ref{config_cfg_man_extension}}{}
\item \contentsline{section}{MAN\_\-LINKS}{\ref{config_cfg_man_links}}{}
\item \contentsline{section}{MAN\_\-OUTPUT}{\ref{config_cfg_man_output}}{}
\item \contentsline{section}{MAX\_\-DOT\_\-GRAPH\_\-DEPTH}{\ref{config_cfg_max_dot_graph_depth}}{}
\item \contentsline{section}{MAX\_\-INITIALIZER\_\-LINES}{\ref{config_cfg_max_initializer_lines}}{}
\item \contentsline{section}{MSCGEN\_\-PATH}{\ref{config_cfg_mscgen_path}}{}
\item \contentsline{section}{MULTILINE\_\-CPP\_\-IS\_\-BRIEF}{\ref{config_cfg_multiline_cpp_is_brief}}{}
\item \contentsline{section}{OPTIMIZE\_\-FOR\_\-FORTRAN}{\ref{config_cfg_optimize_for_fortran}}{}
\item \contentsline{section}{OPTIMIZE\_\-OUTPUT\_\-FOR\_\-C}{\ref{config_cfg_optimize_output_for_c}}{}
\item \contentsline{section}{OPTIMIZE\_\-OUTPUT\_\-JAVA}{\ref{config_cfg_optimize_output_java}}{}
\item \contentsline{section}{OPTIMIZE\_\-OUTPUT\_\-VHDL}{\ref{config_cfg_optimize_output_vhdl}}{}
\item \contentsline{section}{OUTPUT\_\-DIRECTORY}{\ref{config_cfg_output_directory}}{}
\item \contentsline{section}{OUTPUT\_\-LANGUAGE}{\ref{config_cfg_output_language}}{}
\item \contentsline{section}{PAPER\_\-TYPE}{\ref{config_cfg_paper_type}}{}
\item \contentsline{section}{PDF\_\-HYPERLINKS}{\ref{config_cfg_pdf_hyperlinks}}{}
\item \contentsline{section}{PERL\_\-PATH}{\ref{config_cfg_perl_path}}{}
\item \contentsline{section}{PERLMOD\_\-LATEX}{\ref{config_cfg_perlmod_latex}}{}
\item \contentsline{section}{PERLMOD\_\-PRETTY}{\ref{config_cfg_perlmod_pretty}}{}
\item \contentsline{section}{PERLMOD\_\-MAKEVAR\_\-PREFIX}{\ref{config_cfg_perlmod_makevar_prefix}}{}
\item \contentsline{section}{PREDEFINED}{\ref{config_cfg_predefined}}{}
\item \contentsline{section}{PROJECT\_\-NAME}{\ref{config_cfg_project_name}}{}
\item \contentsline{section}{PROJECT\_\-NUMBER}{\ref{config_cfg_project_number}}{}
\item \contentsline{section}{QT\_\-AUTOBRIEF}{\ref{config_cfg_qt_autobrief}}{}
\item \contentsline{section}{QUIET}{\ref{config_cfg_quiet}}{}
\item \contentsline{section}{RECURSIVE}{\ref{config_cfg_recursive}}{}
\item \contentsline{section}{REFERENCED\_\-BY\_\-RELATION}{\ref{config_cfg_referenced_by_relation}}{}
\item \contentsline{section}{REFERENCES\_\-RELATION}{\ref{config_cfg_references_relation}}{}
\item \contentsline{section}{REFERENCES\_\-LINK\_\-SOURCE}{\ref{config_cfg_references_link_source}}{}
\item \contentsline{section}{REPEAT\_\-BRIEF}{\ref{config_cfg_repeat_brief}}{}
\item \contentsline{section}{RTF\_\-EXTENSIONS\_\-FILE}{\ref{config_cfg_rtf_extensions_file}}{}
\item \contentsline{section}{RTF\_\-HYPERLINKS}{\ref{config_cfg_rtf_hyperlinks}}{}
\item \contentsline{section}{RTF\_\-OUTPUT}{\ref{config_cfg_rtf_output}}{}
\item \contentsline{section}{RTF\_\-STYLESHEET\_\-FILE}{\ref{config_cfg_rtf_stylesheet_file}}{}
\item \contentsline{section}{SEARCH\_\-INCLUDES}{\ref{config_cfg_search_includes}}{}
\item \contentsline{section}{SEARCHENGINE}{\ref{config_cfg_searchengine}}{}
\item \contentsline{section}{SEPARATE\_\-MEMBER\_\-PAGES}{\ref{config_cfg_separate_member_pages}}{}
\item \contentsline{section}{SHORT\_\-NAMES}{\ref{config_cfg_short_names}}{}
\item \contentsline{section}{SHOW\_\-DIRECTORIES}{\ref{config_cfg_show_dirs}}{}
\item \contentsline{section}{SHOW\_\-INCLUDE\_\-FILES}{\ref{config_cfg_show_include_files}}{}
\item \contentsline{section}{SHOW\_\-USED\_\-FILES}{\ref{config_cfg_show_used_files}}{}
\item \contentsline{section}{SIP\_\-SUPPORT}{\ref{config_cfg_sip_support}}{}
\item \contentsline{section}{SKIP\_\-FUNCTION\_\-MACROS}{\ref{config_cfg_skip_function_macros}}{}
\item \contentsline{section}{SORT\_\-BRIEF\_\-DOCS}{\ref{config_cfg_sort_brief_docs}}{}
\item \contentsline{section}{SORT\_\-BY\_\-SCOPE\_\-NAME}{\ref{config_cfg_sort_by_scope_name}}{}
\item \contentsline{section}{SORT\_\-GROUP\_\-NAMES}{\ref{config_cfg_sort_group_names}}{}
\item \contentsline{section}{SORT\_\-MEMBER\_\-DOCS}{\ref{config_cfg_sort_member_docs}}{}
\item \contentsline{section}{SOURCE\_\-BROWSER}{\ref{config_cfg_source_browser}}{}
\item \contentsline{section}{STRIP\_\-CODE\_\-COMMENTS}{\ref{config_cfg_strip_code_comments}}{}
\item \contentsline{section}{STRIP\_\-FROM\_\-INC\_\-PATH}{\ref{config_cfg_strip_from_inc_path}}{}
\item \contentsline{section}{STRIP\_\-FROM\_\-PATH}{\ref{config_cfg_strip_from_path}}{}
\item \contentsline{section}{SUBGROUPING}{\ref{config_cfg_subgrouping}}{}
\item \contentsline{section}{TAB\_\-SIZE}{\ref{config_cfg_tab_size}}{}
\item \contentsline{section}{TAGFILES}{\ref{config_cfg_tagfiles}}{}
\item \contentsline{section}{TEMPLATE\_\-RELATIONS}{\ref{config_cfg_template_relations}}{}
\item \contentsline{section}{TOC\_\-EXPAND}{\ref{config_cfg_toc_expand}}{}
\item \contentsline{section}{TREEVIEW\_\-WIDTH}{\ref{config_cfg_treeview_width}}{}
\item \contentsline{section}{TYPEDEF\_\-HIDES\_\-STRUCT}{\ref{config_cfg_typedef_hides_struct}}{}
\item \contentsline{section}{UML\_\-LOOK}{\ref{config_cfg_uml_look}}{}
\item \contentsline{section}{USE\_\-HTAGS}{\ref{config_cfg_use_htags}}{}
\item \contentsline{section}{USE\_\-PDFLATEX}{\ref{config_cfg_use_pdflatex}}{}
\item \contentsline{section}{USE\_\-WINDOWS\_\-ENCODING}{\ref{config_cfg_use_windows_encoding}}{}
\item \contentsline{section}{VERBATIM\_\-HEADERS}{\ref{config_cfg_verbatim_headers}}{}
\item \contentsline{section}{WARN\_\-FORMAT}{\ref{config_cfg_warn_format}}{}
\item \contentsline{section}{WARN\_\-IF\_\-DOC\_\-ERROR}{\ref{config_cfg_warn_if_doc_error}}{}
\item \contentsline{section}{WARN\_\-IF\_\-UNDOCUMENTED}{\ref{config_cfg_warn_if_undocumented}}{}
\item \contentsline{section}{WARN\_\-LOGFILE}{\ref{config_cfg_warn_logfile}}{}
\item \contentsline{section}{WARN\_\-NO\_\-PARAMDOC}{\ref{config_cfg_warn_no_paramdoc}}{}
\item \contentsline{section}{WARNINGS}{\ref{config_cfg_warnings}}{}
\item \contentsline{section}{XML\_\-DTD}{\ref{config_cfg_xml_dtd}}{}
\item \contentsline{section}{XML\_\-OUTPUT}{\ref{config_cfg_xml_output}}{}
\item \contentsline{section}{XML\_\-PROGRAMLISTING}{\ref{config_cfg_xml_programlisting}}{}
\item \contentsline{section}{XML\_\-SCHEMA}{\ref{config_cfg_xml_schema}}{}
\end{CompactList}
\end{multicols}
\normalsize
\hypertarget{config_config_project}{}\section{Project related options}\label{config_config_project}
\label{config_cfg_doxyfile_encoding}
\hypertarget{config_cfg_doxyfile_encoding}{}
 \begin{description}
\item[{\tt DOXYFILE\_\-ENCODING} ]\index{DOXYFILE_ENCODING@{DOXYFILE\_\-ENCODING}} This tag specifies the encoding used for all characters in the config file that follow. The default is UTF-8 which is also the encoding used for all text before the first occurrence of this tag. Doxygen uses libiconv (or the iconv built into libc) for the transcoding. See \href{http://www.gnu.org/software/libiconv}{\tt http://www.gnu.org/software/libiconv} for the list of possible encodings.

\label{config_cfg_project_name}
\hypertarget{config_cfg_project_name}{}
 \item[{\tt PROJECT\_\-NAME} ]\index{PROJECT_NAME@{PROJECT\_\-NAME}} The {\tt PROJECT\_\-NAME} tag is a single word (or a sequence of words surrounded by double-quotes) that should identify the project for which the documentation is generated. This name is used in the title of most generated pages and in a few other places.

\label{config_cfg_project_number}
\hypertarget{config_cfg_project_number}{}
 \item[{\tt PROJECT\_\-NUMBER} ]\index{PROJECT_NUMBER@{PROJECT\_\-NUMBER}} The {\tt PROJECT\_\-NUMBER} tag can be used to enter a project or revision number. This could be handy for archiving the generated documentation or if some version control system is used.

\label{config_cfg_output_directory}
\hypertarget{config_cfg_output_directory}{}
 \item[{\tt OUTPUT\_\-DIRECTORY} ]\index{OUTPUT_DIRECTORY@{OUTPUT\_\-DIRECTORY}} The {\tt OUTPUT\_\-DIRECTORY} tag is used to specify the (relative or absolute) path into which the generated documentation will be written. If a relative path is entered, it will be relative to the location where doxygen was started. If left blank the current directory will be used.

\label{config_cfg_create_subdirs}
\hypertarget{config_cfg_create_subdirs}{}
 \item[{\tt CREATE\_\-SUBDIRS} ]\index{CREATE_SUBDIRS@{CREATE\_\-SUBDIRS}} If the {\tt CREATE\_\-SUBDIRS} tag is set to {\tt YES}, then doxygen will create 4096 sub-directories (in 2 levels) under the output directory of each output format and will distribute the generated files over these directories. Enabling this option can be useful when feeding doxygen a huge amount of source files, where putting all generated files in the same directory would otherwise causes performance problems for the file system.

\label{config_cfg_output_language}
\hypertarget{config_cfg_output_language}{}
 \item[{\tt OUTPUT\_\-LANGUAGE} ]\index{OUTPUT_LANGUAGE@{OUTPUT\_\-LANGUAGE}} The {\tt OUTPUT\_\-LANGUAGE} tag is used to specify the language in which all documentation generated by doxygen is written. Doxygen will use this information to generate all constant output in the proper language. The default language is English, other supported languages are: Afrikaans, Arabic, Brazilian, Catalan, Chinese, Croatian, Czech, Danish, Dutch, Finnish, French, German, Greek, Hungarian, Italian, Japanese, Korean, Lithuanian, Norwegian, Persian, Polish, Portuguese, Romanian, Russian, Serbian, Slovak, Slovene, Spanish, Swedish, and Ukrainian.

\label{config_cfg_use_windows_encoding}
\hypertarget{config_cfg_use_windows_encoding}{}
 \item[{\tt USE\_\-WINDOWS\_\-ENCODING} ]\index{USE_WINDOWS_ENCODING@{USE\_\-WINDOWS\_\-ENCODING}} This tag can be used to specify the encoding used in the generated output. The encoding is not always determined by the language that is chosen, but also whether or not the output is meant for Windows or non-Windows users. In case there is a difference, setting the {\tt USE\_\-WINDOWS\_\-ENCODING} tag to {\tt YES} forces the Windows encoding, (this is the default for the Windows binary), whereas setting the tag to {\tt NO} uses a Unix-style encoding (the default for all platforms other than Windows).

\label{config_cfg_brief_member_desc}
\hypertarget{config_cfg_brief_member_desc}{}
 \item[{\tt BRIEF\_\-MEMBER\_\-DESC} ]\index{BRIEF_MEMBER_DESC@{BRIEF\_\-MEMBER\_\-DESC}} If the {\tt BRIEF\_\-MEMBER\_\-DESC} tag is set to {\tt YES} (the default) doxygen will include brief member descriptions after the members that are listed in the file and class documentation (similar to JavaDoc). Set to NO to disable this.

\label{config_cfg_repeat_brief}
\hypertarget{config_cfg_repeat_brief}{}
 \item[{\tt REPEAT\_\-BRIEF} ]\index{REPEAT_BRIEF@{REPEAT\_\-BRIEF}} If the {\tt REPEAT\_\-BRIEF} tag is set to {\tt YES} (the default) doxygen will prepend the brief description of a member or function before the detailed description

\begin{Desc}
\item[Note: ]If both {\tt HIDE\_\-UNDOC\_\-MEMBERS} and {\tt BRIEF\_\-MEMBER\_\-DESC} are set to {\tt NO}, the brief descriptions will be completely suppressed.\end{Desc}
\label{config_cfg_abbreviate_brief}
\hypertarget{config_cfg_abbreviate_brief}{}
 \item[{\tt ABBREVIATE\_\-BRIEF} ]\index{ABBREVIATE_BRIEF@{ABBREVIATE\_\-BRIEF}} This tag implements a quasi-intelligent brief description abbreviator that is used to form the text in various listings. Each string in this list, if found as the leading text of the brief description, will be stripped from the text and the result after processing the whole list, is used as the annotated text. Otherwise, the brief description is used as-is. If left blank, the following values are used (\char`\"{}$\backslash$\$name\char`\"{} is automatically replaced with the name of the entity): \char`\"{}The \$name class\char`\"{} \char`\"{}The \$name widget\char`\"{} \char`\"{}The \$name file\char`\"{} \char`\"{}is\char`\"{} \char`\"{}provides\char`\"{} \char`\"{}specifies\char`\"{} \char`\"{}contains\char`\"{} \char`\"{}represents\char`\"{} \char`\"{}a\char`\"{} \char`\"{}an\char`\"{} \char`\"{}the\char`\"{}.

\label{config_cfg_always_detailed_sec}
\hypertarget{config_cfg_always_detailed_sec}{}
 \item[{\tt ALWAYS\_\-DETAILED\_\-SEC} ]\index{ALWAYS_DETAILED_SEC@{ALWAYS\_\-DETAILED\_\-SEC}} If the {\tt ALWAYS\_\-DETAILED\_\-SEC} and {\tt REPEAT\_\-BRIEF} tags are both set to {\tt YES} then doxygen will generate a detailed section even if there is only a brief description.

\label{config_cfg_inline_inherited_memb}
\hypertarget{config_cfg_inline_inherited_memb}{}
 \item[{\tt INLINE\_\-INHERITED\_\-MEMB} ]\index{INLINE_INHERITED_MEMB@{INLINE\_\-INHERITED\_\-MEMB}} If the {\tt INLINE\_\-INHERITED\_\-MEMB} tag is set to {\tt YES}, doxygen will show all inherited members of a class in the documentation of that class as if those members were ordinary class members. Constructors, destructors and assignment operators of the base classes will not be shown.

\label{config_cfg_full_path_names}
\hypertarget{config_cfg_full_path_names}{}
 \item[{\tt FULL\_\-PATH\_\-NAMES} ]\index{FULL_PATH_NAMES@{FULL\_\-PATH\_\-NAMES}} If the {\tt FULL\_\-PATH\_\-NAMES} tag is set to {\tt YES} doxygen will prepend the full path before files name in the file list and in the header files. If set to NO the shortest path that makes the file name unique will be used

\label{config_cfg_strip_from_path}
\hypertarget{config_cfg_strip_from_path}{}
 \item[{\tt STRIP\_\-FROM\_\-PATH} ]\index{STRIP_FROM_PATH@{STRIP\_\-FROM\_\-PATH}} If the {\tt FULL\_\-PATH\_\-NAMES} tag is set to {\tt YES} then the {\tt STRIP\_\-FROM\_\-PATH} tag can be used to strip a user-defined part of the path. Stripping is only done if one of the specified strings matches the left-hand part of the path. The tag can be used to show relative paths in the file list. If left blank the directory from which doxygen is run is used as the path to strip.

\label{config_cfg_strip_from_inc_path}
\hypertarget{config_cfg_strip_from_inc_path}{}
 \item[{\tt STRIP\_\-FROM\_\-INC\_\-PATH} ]\index{STRIP_FROM_INC_PATH@{STRIP\_\-FROM\_\-INC\_\-PATH}} The {\tt STRIP\_\-FROM\_\-INC\_\-PATH} tag can be used to strip a user-defined part of the path mentioned in the documentation of a class, which tells the reader which header file to include in order to use a class. If left blank only the name of the header file containing the class definition is used. Otherwise one should specify the include paths that are normally passed to the compiler using the -I flag.

\label{config_cfg_case_sense_names}
\hypertarget{config_cfg_case_sense_names}{}
 \item[{\tt CASE\_\-SENSE\_\-NAMES} ]\index{CASE_SENSE_NAMES@{CASE\_\-SENSE\_\-NAMES}} If the {\tt CASE\_\-SENSE\_\-NAMES} tag is set to {\tt NO} then doxygen will only generate file names in lower-case letters. If set to {\tt YES} upper-case letters are also allowed. This is useful if you have classes or files whose names only differ in case and if your file system supports case sensitive file names. Windows users are advised to set this option to NO.

\label{config_cfg_short_names}
\hypertarget{config_cfg_short_names}{}
 \item[{\tt SHORT\_\-NAMES} ]\index{SHORT_NAMES@{SHORT\_\-NAMES}} If the {\tt SHORT\_\-NAMES} tag is set to {\tt YES}, doxygen will generate much shorter (but less readable) file names. This can be useful is your file systems doesn't support long names like on DOS, Mac, or CD-ROM.

\label{config_cfg_javadoc_autobrief}
\hypertarget{config_cfg_javadoc_autobrief}{}
 \item[{\tt JAVADOC\_\-AUTOBRIEF} ]\index{JAVADOC_AUTOBRIEF@{JAVADOC\_\-AUTOBRIEF}} If the {\tt JAVADOC\_\-AUTOBRIEF} is set to {\tt YES} then doxygen will interpret the first line (until the first dot) of a JavaDoc-style comment as the brief description. If set to NO (the default), the Javadoc-style will behave just like regular Qt-style comments (thus requiring an explicit @brief command for a brief description.)

\label{config_cfg_qt_autobrief}
\hypertarget{config_cfg_qt_autobrief}{}
 \item[{\tt QT\_\-AUTOBRIEF} ]\index{QT_AUTOBRIEF@{QT\_\-AUTOBRIEF}} If the {\tt QT\_\-AUTOBRIEF} is set to {\tt YES} then doxygen will interpret the first line (until the first dot) of a Qt-style comment as the brief description. If set to NO (the default), the Qt-style will behave just like regular Qt-style comments (thus requiring an explicit $\backslash$brief command for a brief description.)

\label{config_cfg_builtin_stl_support}
\hypertarget{config_cfg_builtin_stl_support}{}
 \item[{\tt BUILTIN\_\-STL\_\-SUPPORT} ]\index{BUILTIN_STL_SUPPORT@{BUILTIN\_\-STL\_\-SUPPORT}} If you use STL classes (i.e. std::string, std::vector, etc.) but do not want to include (a tag file for) the STL sources as input, then you should set this tag to {\tt YES} in order to let doxygen match functions declarations and definitions whose arguments contain STL classes (e.g. func(std::string); v.s. func(std::string) \{\}). This also make the inheritance and collaboration diagrams that involve STL classes more complete and accurate.

\label{config_cfg_cpp_cli_support}
\hypertarget{config_cfg_cpp_cli_support}{}
 \item[{\tt CPP\_\-CLI\_\-SUPPORT} ]\index{CPP_CLI_SUPPORT@{CPP\_\-CLI\_\-SUPPORT}} If you use Microsoft's C++/CLI language, you should set this option to YES to enable parsing support.

\label{config_cfg_sip_support}
\hypertarget{config_cfg_sip_support}{}
 \item[{\tt SIP\_\-SUPPORT} ]\index{OPTIMIZE_OUTPUT_SIP@{OPTIMIZE\_\-OUTPUT\_\-SIP}} Set the SIP\_\-SUPPORT tag to YES if your project consists of \href{http://www.riverbankcomputing.co.uk/sip/}{\tt sip} sources only. Doxygen will parse them like normal C++ but will assume all classes use public instead of private inheritance when no explicit protection keyword is present.

\label{config_cfg_distribute_group_doc}
\hypertarget{config_cfg_distribute_group_doc}{}
 \item[{\tt DISTRIBUTE\_\-GROUP\_\-DOC} ]\index{DISTRIBUTE_GROUP_DOC@{DISTRIBUTE\_\-GROUP\_\-DOC}} If member grouping is used in the documentation and the DISTRIBUTE\_\-GROUP\_\-DOC tag is set to YES, then doxygen will reuse the documentation of the first member in the group (if any) for the other members of the group. By default all members of a group must be documented explicitly.

\label{config_cfg_multiline_cpp_is_brief}
\hypertarget{config_cfg_multiline_cpp_is_brief}{}
 \item[{\tt MULTILINE\_\-CPP\_\-IS\_\-BRIEF} ]\index{MULTILINE_CPP_IS_BRIEF@{MULTILINE\_\-CPP\_\-IS\_\-BRIEF}} The MULTILINE\_\-CPP\_\-IS\_\-BRIEF tag can be set to YES to make Doxygen treat a multi-line C++ special comment block (i.e. a block of //! or /// comments) as a brief description. This used to be the default behaviour. The new default is to treat a multi-line C++ comment block as a detailed description. Set this tag to YES if you prefer the old behaviour instead. Note that setting this tag to YES also means that rational rose comments are not recognized any more.

\label{config_cfg_details_at_top}
\hypertarget{config_cfg_details_at_top}{}
 \item[{\tt DETAILS\_\-AT\_\-TOP} ]\index{DETAILS_AT_TOP@{DETAILS\_\-AT\_\-TOP}} If the DETAILS\_\-AT\_\-TOP tag is set to YES then Doxygen will output the detailed description near the top, like JavaDoc. If set to NO, the detailed description appears after the member documentation.

\label{config_cfg_inherit_docs}
\hypertarget{config_cfg_inherit_docs}{}
 \item[{\tt INHERIT\_\-DOCS} ]\index{INHERIT_DOCS@{INHERIT\_\-DOCS}} If the {\tt INHERIT\_\-DOCS} tag is set to {\tt YES} (the default) then an undocumented member inherits the documentation from any documented member that it re-implements.

\label{config_cfg_separate_member_pages}
\hypertarget{config_cfg_separate_member_pages}{}
 \item[{\tt SEPARATE\_\-MEMBER\_\-PAGES} ]\index{SEPARATE_MEMBER_PAGES@{SEPARATE\_\-MEMBER\_\-PAGES}} If the {\tt SEPARATE\_\-MEMBER\_\-PAGES} tag is set to {\tt YES}, then doxygen will produce a new page for each member. If set to {\tt NO}, the documentation of a member will be part of the file/class/namespace that contains it.

\label{config_cfg_tab_size}
\hypertarget{config_cfg_tab_size}{}
 \item[{\tt TAB\_\-SIZE} ]\index{TAB_SIZE@{TAB\_\-SIZE}} the {\tt TAB\_\-SIZE} tag can be used to set the number of spaces in a tab. Doxygen uses this value to replace tabs by spaces in code fragments.

\label{config_cfg_aliases}
\hypertarget{config_cfg_aliases}{}
 \item[{\tt ALIASES} ]\index{ALIASES@{ALIASES}} This tag can be used to specify a number of aliases that acts as commands in the documentation. An alias has the form 

\footnotesize\begin{verbatim}
 name=value
\end{verbatim}
\normalsize
 For example adding 

\footnotesize\begin{verbatim}
 "sideeffect=\par Side Effects:\n" 
\end{verbatim}
\normalsize
 will allow you to put the command $\backslash$sideeffect (or @sideeffect) in the documentation, which will result in a user-defined paragraph with heading \char`\"{}Side Effects:\char`\"{}. You can put $\backslash$n's in the value part of an alias to insert newlines.

\label{config_cfg_optimize_output_for_c}
\hypertarget{config_cfg_optimize_output_for_c}{}
 \item[{\tt OPTIMIZE\_\-OUTPUT\_\-FOR\_\-C} ]\index{OPTIMIZE_OUTPUT_FOR_C@{OPTIMIZE\_\-OUTPUT\_\-FOR\_\-C}} Set the {\tt OPTIMIZE\_\-OUTPUT\_\-FOR\_\-C} tag to {\tt YES} if your project consists of C sources only. Doxygen will then generate output that is more tailored for C. For instance, some of the names that are used will be different. The list of all members will be omitted, etc.

\label{config_cfg_optimize_output_java}
\hypertarget{config_cfg_optimize_output_java}{}
 \item[{\tt OPTIMIZE\_\-OUTPUT\_\-JAVA} ]\index{OPTIMIZE_OUTPUT_JAVA@{OPTIMIZE\_\-OUTPUT\_\-JAVA}} Set the OPTIMIZE\_\-OUTPUT\_\-JAVA tag to YES if your project consists of Java or Python sources only. Doxygen will then generate output that is more tailored for that language. For instance, namespaces will be presented as packages, qualified scopes will look different, etc.

\label{config_cfg_optimize_for_fortran}
\hypertarget{config_cfg_optimize_for_fortran}{}
 \item[{\tt OPTIMIZE\_\-FOR\_\-FORTRAN} ]\index{OPTIMIZE_FOR_FORTRAN@{OPTIMIZE\_\-FOR\_\-FORTRAN}} Set the {\tt OPTIMIZE\_\-FOR\_\-FORTRAN} tag to {\tt YES} if your project consists of Fortran sources. Doxygen will then generate output that is tailored for Fortran.

\label{config_cfg_optimize_output_vhdl}
\hypertarget{config_cfg_optimize_output_vhdl}{}
 \item[{\tt OPTIMIZE\_\-OUTPUT\_\-VHDL} ]\index{OPTIMIZE_OUTPUT_VHDL@{OPTIMIZE\_\-OUTPUT\_\-VHDL}} Set the {\tt OPTIMIZE\_\-OUTPUT\_\-VHDL} tag to {\tt YES} if your project consists of VHDL sources. Doxygen will then generate output that is tailored for VHDL.

\label{config_cfg_subgrouping}
\hypertarget{config_cfg_subgrouping}{}
 \item[{\tt SUBGROUPING} ]\index{SUBGROUPING@{SUBGROUPING}} Set the {\tt SUBGROUPING} tag to {\tt YES} (the default) to allow class member groups of the same type (for instance a group of public functions) to be put as a subgroup of that type (e.g. under the Public Functions section). Set it to {\tt NO} to prevent subgrouping. Alternatively, this can be done per class using the \hyperlink{commands_cmdnosubgrouping}{$\backslash$nosubgrouping} command.

\label{config_cfg_typedef_hides_struct}
\hypertarget{config_cfg_typedef_hides_struct}{}
 \item[{\tt TYPEDEF\_\-HIDES\_\-STRUCT} ]\index{TYPEDEF_HIDES_STRUCT@{TYPEDEF\_\-HIDES\_\-STRUCT}} When {\tt TYPEDEF\_\-HIDES\_\-STRUCT} is enabled, a typedef of a struct, union, or enum is documented as struct, union, or enum with the name of the typedef. So {\tt typedef struct TypeS \{\} TypeT}, will appear in the documentation as a struct with name {\tt TypeT}. When disabled the typedef will appear as a member of a file, namespace, or class. And the struct will be named {\tt TypeS}. This can typically be useful for C code in case the coding convention dictates that all compound types are typedef'ed and only the typedef is referenced, never the tag name.

\end{description}
\hypertarget{config_config_build}{}\section{Build related options}\label{config_config_build}
\label{config_cfg_extract_all}
\hypertarget{config_cfg_extract_all}{}
 \begin{description}
\item[{\tt EXTRACT\_\-ALL} ]\index{EXTRACT_ALL@{EXTRACT\_\-ALL}} If the {\tt EXTRACT\_\-ALL} tag is set to {\tt YES} doxygen will assume all entities in documentation are documented, even if no documentation was available. Private class members and static file members will be hidden unless the {\tt EXTRACT\_\-PRIVATE} and {\tt EXTRACT\_\-STATIC} tags are set to {\tt YES} 

\begin{Desc}
\item[Note: ]This will also disable the warnings about undocumented members that are normally produced when {\tt WARNINGS} is set to {\tt YES} \end{Desc}
\label{config_cfg_extract_private}
\hypertarget{config_cfg_extract_private}{}
 \item[{\tt EXTRACT\_\-PRIVATE} ]\index{EXTRACT_PRIVATE@{EXTRACT\_\-PRIVATE}} If the {\tt EXTRACT\_\-PRIVATE} tag is set to {\tt YES} all private members of a class will be included in the documentation.

\label{config_cfg_extract_static}
\hypertarget{config_cfg_extract_static}{}
 \item[{\tt EXTRACT\_\-STATIC} ]\index{EXTRACT_STATIC@{EXTRACT\_\-STATIC}} If the {\tt EXTRACT\_\-STATIC} tag is set to {\tt YES} all static members of a file will be included in the documentation.

\label{config_cfg_extract_local_classes}
\hypertarget{config_cfg_extract_local_classes}{}
 \item[{\tt EXTRACT\_\-LOCAL\_\-CLASSES} ]\index{EXTRACT_LOCAL_CLASSES@{EXTRACT\_\-LOCAL\_\-CLASSES}} If the {\tt EXTRACT\_\-LOCAL\_\-CLASSES} tag is set to {\tt YES} classes (and structs) defined locally in source files will be included in the documentation. If set to NO only classes defined in header files are included. Does not have any effect for Java sources.

\label{config_cfg_extract_anon_nspaces}
\hypertarget{config_cfg_extract_anon_nspaces}{}
 \item[{\tt EXTRACT\_\-ANON\_\-NSPACES} ]\index{EXTRACT_ANON_NSPACES@{EXTRACT\_\-ANON\_\-NSPACES}} If this flag is set to YES, the members of anonymous namespaces will be extracted and appear in the documentation as a namespace called 'anonymous\_\-namespace\{file\}', where file will be replaced with the base name of the file that contains the anonymous namespace. By default anonymous namespace are hidden.

\label{config_cfg_extract_local_methods}
\hypertarget{config_cfg_extract_local_methods}{}
 \item[{\tt EXTRACT\_\-LOCAL\_\-METHODS} ]\index{EXTRACT_LOCAL_METHODS@{EXTRACT\_\-LOCAL\_\-METHODS}} This flag is only useful for Objective-C code. When set to {\tt YES} local methods, which are defined in the implementation section but not in the interface are included in the documentation. If set to {\tt NO} (the default) only methods in the interface are included.

\label{config_cfg_hide_undoc_members}
\hypertarget{config_cfg_hide_undoc_members}{}
 \item[{\tt HIDE\_\-UNDOC\_\-MEMBERS} ]\index{HIDE_UNDOC_MEMBERS@{HIDE\_\-UNDOC\_\-MEMBERS}} If the {\tt HIDE\_\-UNDOC\_\-MEMBERS} tag is set to {\tt YES}, doxygen will hide all undocumented members inside documented classes or files. If set to {\tt NO} (the default) these members will be included in the various overviews, but no documentation section is generated. This option has no effect if {\tt EXTRACT\_\-ALL} is enabled.

\label{config_cfg_hide_undoc_classes}
\hypertarget{config_cfg_hide_undoc_classes}{}
 \item[{\tt HIDE\_\-UNDOC\_\-CLASSES} ]\index{HIDE_UNDOC_CLASSES@{HIDE\_\-UNDOC\_\-CLASSES}} If the {\tt HIDE\_\-UNDOC\_\-CLASSESS} tag is set to {\tt YES}, doxygen will hide all undocumented classes. If set to {\tt NO} (the default) these classes will be included in the various overviews. This option has no effect if {\tt EXTRACT\_\-ALL} is enabled.

\label{config_cfg_hide_friend_compounds}
\hypertarget{config_cfg_hide_friend_compounds}{}
 \item[{\tt HIDE\_\-FRIEND\_\-COMPOUNDS} ]\index{HIDE_FRIEND_COMPOUNDS@{HIDE\_\-FRIEND\_\-COMPOUNDS}} If the {\tt HIDE\_\-FRIEND\_\-COMPOUNDS} tag is set to {\tt YES}, Doxygen will hide all friend (class$|$struct$|$union) declarations. If set to {\tt NO} (the default) these declarations will be included in the documentation.

\label{config_cfg_hide_in_body_docs}
\hypertarget{config_cfg_hide_in_body_docs}{}
 \item[{\tt HIDE\_\-IN\_\-BODY\_\-DOCS} ]\index{HIDE_IN_BODY_DOCS@{HIDE\_\-IN\_\-BODY\_\-DOCS}} If the {\tt HIDE\_\-IN\_\-BODY\_\-DOCS} tag is set to {\tt YES}, Doxygen will hide any documentation blocks found inside the body of a function. If set to {\tt NO} (the default) these blocks will be appended to the function's detailed documentation block.

\label{config_cfg_internal_docs}
\hypertarget{config_cfg_internal_docs}{}
 \item[{\tt INTERNAL\_\-DOCS} ]\index{INTERNAL_DOCS@{INTERNAL\_\-DOCS}} The {\tt INTERNAL\_\-DOCS} tag determines if documentation that is typed after a \hyperlink{commands_cmdinternal}{$\backslash$internal} command is included. If the tag is set to {\tt NO} (the default) then the documentation will be excluded. Set it to {\tt YES} to include the internal documentation.

\label{config_cfg_hide_scope_names}
\hypertarget{config_cfg_hide_scope_names}{}
 \item[{\tt HIDE\_\-SCOPE\_\-NAMES} ]\index{HIDE_SCOPE_NAMES@{HIDE\_\-SCOPE\_\-NAMES}} If the {\tt HIDE\_\-SCOPE\_\-NAMES} tag is set to {\tt NO} (the default) then doxygen will show members with their full class and namespace scopes in the documentation. If set to {\tt YES} the scope will be hidden.

\label{config_cfg_show_include_files}
\hypertarget{config_cfg_show_include_files}{}
 \item[{\tt SHOW\_\-INCLUDE\_\-FILES} ]\index{SHOW_INCLUDE_FILES@{SHOW\_\-INCLUDE\_\-FILES}} If the SHOW\_\-INCLUDE\_\-FILES tag is set to YES (the default) then doxygen will put a list of the files that are included by a file in the documentation of that file.

\label{config_cfg_inline_info}
\hypertarget{config_cfg_inline_info}{}
 \item[{\tt INLINE\_\-INFO} ]\index{INLINE_INFO @{INLINE\_\-INFO }} If the {\tt INLINE\_\-INFO} tag is set to {\tt YES} (the default) then a tag \mbox{[}inline\mbox{]} is inserted in the documentation for inline members.

\label{config_cfg_sort_member_docs}
\hypertarget{config_cfg_sort_member_docs}{}
 \item[{\tt SORT\_\-MEMBER\_\-DOCS} ]\index{SORT_MEMBER_DOCS@{SORT\_\-MEMBER\_\-DOCS}} If the {\tt SORT\_\-MEMBER\_\-DOCS} tag is set to {\tt YES} (the default) then doxygen will sort the (detailed) documentation of file and class members alphabetically by member name. If set to {\tt NO} the members will appear in declaration order.

\label{config_cfg_sort_brief_docs}
\hypertarget{config_cfg_sort_brief_docs}{}
 \item[{\tt SORT\_\-BRIEF\_\-DOCS} ]\index{SORT_BRIEF_DOCS@{SORT\_\-BRIEF\_\-DOCS}} If the {\tt SORT\_\-BRIEF\_\-DOCS} tag is set to {\tt YES} then doxygen will sort the brief descriptions of file, namespace and class members alphabetically by member name. If set to {\tt NO} (the default) the members will appear in declaration order.

\label{config_cfg_sort_group_names}
\hypertarget{config_cfg_sort_group_names}{}
 \item[{\tt SORT\_\-GROUP\_\-NAMES} ]\index{SORT_GROUP_NAMES@{SORT\_\-GROUP\_\-NAMES}} If the {\tt SORT\_\-GROUP\_\-NAMES} tag is set to {\tt YES} then doxygen will sort the hierarchy of group names into alphabetical order. If set to {\tt NO} (the default) the group names will appear in their defined order.

\label{config_cfg_sort_by_scope_name}
\hypertarget{config_cfg_sort_by_scope_name}{}
 \item[{\tt SORT\_\-BY\_\-SCOPE\_\-NAME} ]\index{SORT_BY_SCOPE_NAME@{SORT\_\-BY\_\-SCOPE\_\-NAME}} If the {\tt SORT\_\-BY\_\-SCOPE\_\-NAME} tag is set to {\tt YES}, the class list will be sorted by fully-qualified names, including namespaces. If set to NO (the default), the class list will be sorted only by class name, not including the namespace part. \begin{Desc}
\item[Note:]This option is not very useful if {\tt HIDE\_\-SCOPE\_\-NAMES} is set to {\tt YES}. 

This option applies only to the class list, not to the alphabetical list.\end{Desc}
\label{config_cfg_generate_deprecatedlist}
\hypertarget{config_cfg_generate_deprecatedlist}{}
 \item[{\tt GENERATE\_\-DEPRECATEDLIST} ]\index{GENERATE_DEPRECATEDLIST@{GENERATE\_\-DEPRECATEDLIST}} The GENERATE\_\-DEPRECATEDLIST tag can be used to enable (YES) or disable (NO) the deprecated list. This list is created by putting \hyperlink{commands_cmddeprecated}{$\backslash$deprecated} commands in the documentation.

\label{config_cfg_generate_todolist}
\hypertarget{config_cfg_generate_todolist}{}
 \item[{\tt GENERATE\_\-TODOLIST} ]\index{GENERATE_TODOLIST@{GENERATE\_\-TODOLIST}} The GENERATE\_\-TODOLIST tag can be used to enable (YES) or disable (NO) the todo list. This list is created by putting \hyperlink{commands_cmdtodo}{$\backslash$todo} commands in the documentation.

\label{config_cfg_generate_testlist}
\hypertarget{config_cfg_generate_testlist}{}
 \item[{\tt GENERATE\_\-TESTLIST} ]\index{GENERATE_TESTLIST@{GENERATE\_\-TESTLIST}} The GENERATE\_\-TESTLIST tag can be used to enable (YES) or disable (NO) the test list. This list is created by putting \hyperlink{commands_cmdtest}{$\backslash$test} commands in the documentation.

\label{config_cfg_generate_buglist}
\hypertarget{config_cfg_generate_buglist}{}
 \item[{\tt GENERATE\_\-BUGLIST} ]\index{GENERATE_BUGLIST@{GENERATE\_\-BUGLIST}} The GENERATE\_\-BUGLIST tag can be used to enable (YES) or disable (NO) the bug list. This list is created by putting \hyperlink{commands_cmdbug}{$\backslash$bug} commands in the documentation.

\label{config_cfg_enabled_sections}
\hypertarget{config_cfg_enabled_sections}{}
 \item[{\tt ENABLED\_\-SECTIONS} ]\index{ENABLED_SECTIONS@{ENABLED\_\-SECTIONS}} The {\tt ENABLED\_\-SECTIONS} tag can be used to enable conditional documentation sections, marked by \hyperlink{commands_cmdif}{$\backslash$if} $<$section-label$>$ ... \hyperlink{commands_cmdendif}{$\backslash$endif} and \hyperlink{commands_cmdcond}{$\backslash$cond} $<$section-label$>$ ... \hyperlink{commands_cmdendcond}{$\backslash$endcond} blocks.

\label{config_cfg_max_initializer_lines}
\hypertarget{config_cfg_max_initializer_lines}{}
 \item[{\tt MAX\_\-INITIALIZER\_\-LINES} ]\index{MAX_INITIALIZER_LINES@{MAX\_\-INITIALIZER\_\-LINES}} The {\tt MAX\_\-INITIALIZER\_\-LINES} tag determines the maximum number of lines that the initial value of a variable or define can be. If the initializer consists of more lines than specified here it will be hidden. Use a value of 0 to hide initializers completely. The appearance of the value of individual variables and defines can be controlled using \hyperlink{commands_cmdshowinitializer}{$\backslash$showinitializer} or \hyperlink{commands_cmdhideinitializer}{$\backslash$hideinitializer} command in the documentation.

\label{config_cfg_show_used_files}
\hypertarget{config_cfg_show_used_files}{}
 \item[{\tt SHOW\_\-USED\_\-FILES} ]\index{SHOW_USED_FILES@{SHOW\_\-USED\_\-FILES}} Set the {\tt SHOW\_\-USED\_\-FILES} tag to {\tt NO} to disable the list of files generated at the bottom of the documentation of classes and structs. If set to {\tt YES} the list will mention the files that were used to generate the documentation.

\label{config_cfg_show_dirs}
\hypertarget{config_cfg_show_dirs}{}
 \item[{\tt SHOW\_\-DIRECTORIES} ]\index{SHOW_DIRECTORIES@{SHOW\_\-DIRECTORIES}} If the sources in your project are distributed over multiple directories then setting the SHOW\_\-DIRECTORIES tag to YES will show the directory hierarchy in the documentation.

\end{description}
\hypertarget{config_messages_input}{}\section{Options related to warning and progress messages}\label{config_messages_input}
\label{config_cfg_quiet}
\hypertarget{config_cfg_quiet}{}
 \begin{description}
\item[{\tt QUIET} ]\index{QUIET@{QUIET}} The {\tt QUIET} tag can be used to turn on/off the messages that are generated to standard output by doxygen. Possible values are {\tt YES} and {\tt NO}, where {\tt YES} implies that the messages are off. If left blank {\tt NO} is used.

\label{config_cfg_warnings}
\hypertarget{config_cfg_warnings}{}
 \item[{\tt WARNINGS} ]\index{WARNINGS@{WARNINGS}} The {\tt WARNINGS} tag can be used to turn on/off the warning messages that are generated to standard error by doxygen. Possible values are {\tt YES} and {\tt NO}, where {\tt YES} implies that the warnings are on. If left blank {\tt NO} is used.

{\bf Tip:} Turn warnings on while writing the documentation.

\label{config_cfg_warn_if_undocumented}
\hypertarget{config_cfg_warn_if_undocumented}{}
 \item[{\tt WARN\_\-IF\_\-UNDOCUMENTED} ]\index{WARN_IF_UNDOCUMENTED@{WARN\_\-IF\_\-UNDOCUMENTED}} If {\tt WARN\_\-IF\_\-UNDOCUMENTED} is set to {\tt YES}, then doxygen will generate warnings for undocumented members. If {\tt EXTRACT\_\-ALL} is set to {\tt YES} then this flag will automatically be disabled.

\label{config_cfg_warn_if_doc_error}
\hypertarget{config_cfg_warn_if_doc_error}{}
 \item[{\tt WARN\_\-IF\_\-DOC\_\-ERROR} ]\index{WARN_IF_DOC_ERROR@{WARN\_\-IF\_\-DOC\_\-ERROR}} If {\tt WARN\_\-IF\_\-DOC\_\-ERROR} is set to {\tt YES}, doxygen will generate warnings for potential errors in the documentation, such as not documenting some parameters in a documented function, or documenting parameters that don't exist or using markup commands wrongly.

\label{config_cfg_warn_no_paramdoc}
\hypertarget{config_cfg_warn_no_paramdoc}{}
 \item[{\tt WARN\_\-NO\_\-PARAMDOC} ]\index{WARN_NO_PARAMDOC@{WARN\_\-NO\_\-PARAMDOC}} This {\tt WARN\_\-NO\_\-PARAMDOC} option can be abled to get warnings for functions that are documented, but have no documentation for their parameters or return value. If set to {\tt NO} (the default) doxygen will only warn about wrong or incomplete parameter documentation, but not about the absence of documentation.

\label{config_cfg_warn_format}
\hypertarget{config_cfg_warn_format}{}
 \item[{\tt WARN\_\-FORMAT} ]\index{WARN_FORMAT@{WARN\_\-FORMAT}} The {\tt WARN\_\-FORMAT} tag determines the format of the warning messages that doxygen can produce. The string should contain the {\tt \$file}, {\tt \$line}, and {\tt \$text} tags, which will be replaced by the file and line number from which the warning originated and the warning text.

\label{config_cfg_warn_logfile}
\hypertarget{config_cfg_warn_logfile}{}
 \item[{\tt WARN\_\-LOGFILE} ]\index{WARN_LOGFILE@{WARN\_\-LOGFILE}} The {\tt WARN\_\-LOGFILE} tag can be used to specify a file to which warning and error messages should be written. If left blank the output is written to stderr.

\end{description}
\hypertarget{config_config_input}{}\section{Input related options}\label{config_config_input}
\label{config_cfg_input}
\hypertarget{config_cfg_input}{}
 \begin{description}
\item[{\tt INPUT} ]\index{INPUT@{INPUT}} The {\tt INPUT} tag is used to specify the files and/or directories that contain documented source files. You may enter file names like {\tt myfile.cpp} or directories like {\tt /usr/src/myproject}. Separate the files or directories with spaces.\par


{\bf Note:} If this tag is empty the current directory is searched.

\label{config_cfg_input_encoding}
\hypertarget{config_cfg_input_encoding}{}
 \item[{\tt INPUT\_\-ENCODING} ]\index{INPUT_ENCODING@{INPUT\_\-ENCODING}} This tag can be used to specify the character encoding of the source files that doxygen parses. Internally doxygen uses the UTF-8 encoding, which is also the default input encoding. Doxygen uses libiconv (or the iconv built into libc) for the transcoding. See \href{http://www.gnu.org/software/libiconv}{\tt the libiconv documentation} for the list of possible encodings.

\label{config_cfg_file_patterns}
\hypertarget{config_cfg_file_patterns}{}
 \item[{\tt FILE\_\-PATTERNS} ]\index{FILE_PATTERNS@{FILE\_\-PATTERNS}} If the value of the {\tt INPUT} tag contains directories, you can use the {\tt FILE\_\-PATTERNS} tag to specify one or more wildcard patterns (like {\tt $\ast$}.cpp and {\tt $\ast$}.h ) to filter out the source-files in the directories. If left blank the following patterns are tested: {\tt  .c $\ast$.cc $\ast$.cxx $\ast$.cpp $\ast$.c++ $\ast$.java $\ast$.ii $\ast$.ixx $\ast$.ipp $\ast$.i++ $\ast$.inl $\ast$.h $\ast$.hh $\ast$.hxx $\ast$.hpp .h++ $\ast$.idl $\ast$.odl $\ast$.cs $\ast$.php $\ast$.php3 $\ast$.inc $\ast$.m $\ast$.mm }

\label{config_cfg_file_version_filter}
\hypertarget{config_cfg_file_version_filter}{}
 \item[{\tt FILE\_\-VERSION\_\-FILTER} ]\index{FILE_VERSION_FILTER@{FILE\_\-VERSION\_\-FILTER}} The {\tt FILE\_\-VERSION\_\-FILTER} tag can be used to specify a program or script that doxygen should invoke to get the current version for each file (typically from the version control system). Doxygen will invoke the program by executing (via popen()) the command {\tt command input-file}, where {\tt command} is the value of the {\tt FILE\_\-VERSION\_\-FILTER} tag, and {\tt input-file} is the name of an input file provided by doxygen. Whatever the program writes to standard output is used as the file version.

Example of using a shell script as a filter for Unix: 

\footnotesize\begin{verbatim}
 FILE_VERSION_FILTER = "/bin/sh versionfilter.sh"
\end{verbatim}
\normalsize


Example shell script for CVS: 

\footnotesize\begin{verbatim}
#!/bin/sh
cvs status $1 | sed -n 's/^[ \]*Working revision:[ \t]*\([0-9][0-9\.]*\).*/\1/p'
\end{verbatim}
\normalsize


Example shell script for Subversion: 

\footnotesize\begin{verbatim}
#!/bin/sh
svn stat -v $1 | sed -n 's/^[ A-Z?\*|!]\{1,15\}/r/;s/ \{1,15\}/\/r/;s/ .*//p'
\end{verbatim}
\normalsize


Example filter for ClearCase: 

\footnotesize\begin{verbatim}
FILE_VERSION_INFO = "cleartool desc -fmt \%Vn"
\end{verbatim}
\normalsize


\label{config_cfg_recursive}
\hypertarget{config_cfg_recursive}{}
 \item[{\tt RECURSIVE} ]\index{RECURSIVE@{RECURSIVE}} The {\tt RECURSIVE} tag can be used to specify whether or not subdirectories should be searched for input files as well. Possible values are {\tt YES} and {\tt NO}. If left blank {\tt NO} is used.

\label{config_cfg_exclude}
\hypertarget{config_cfg_exclude}{}
 \item[{\tt EXCLUDE} ]\index{EXCLUDE@{EXCLUDE}} The {\tt EXCLUDE} tag can be used to specify files and/or directories that should excluded from the {\tt INPUT} source files. This way you can easily exclude a subdirectory from a directory tree whose root is specified with the {\tt INPUT} tag.

\label{config_cfg_exclude_symlinks}
\hypertarget{config_cfg_exclude_symlinks}{}
 \item[{\tt EXCLUDE\_\-SYMLINKS} ]\index{EXCLUDE_SYMLINKS@{EXCLUDE\_\-SYMLINKS}} The {\tt EXCLUDE\_\-SYMLINKS} tag can be used select whether or not files or directories that are symbolic links (a Unix filesystem feature) are excluded from the input.

\label{config_cfg_exclude_patterns}
\hypertarget{config_cfg_exclude_patterns}{}
 \item[{\tt EXCLUDE\_\-PATTERNS} ]\index{EXCLUDE_PATTERNS@{EXCLUDE\_\-PATTERNS}} If the value of the {\tt INPUT} tag contains directories, you can use the {\tt EXCLUDE\_\-PATTERNS} tag to specify one or more wildcard patterns to exclude certain files from those directories.

Note that the wildcards are matched against the file with absolute path, so to exclude all test directories use the pattern {\tt $\ast$}{\tt /test/}{\tt $\ast$}

\label{config_cfg_example_path}
\hypertarget{config_cfg_example_path}{}
 \item[{\tt EXAMPLE\_\-PATH} ]\index{EXAMPLE_PATH@{EXAMPLE\_\-PATH}} The {\tt EXAMPLE\_\-PATH} tag can be used to specify one or more files or directories that contain example code fragments that are included (see the $\backslash$include command in section \hyperlink{commands_cmdinclude}{$\backslash$include}).

\label{config_cfg_example_recursive}
\hypertarget{config_cfg_example_recursive}{}
 \item[{\tt EXAMPLE\_\-RECURSIVE} ]\index{EXAMPLE_RECURSIVE@{EXAMPLE\_\-RECURSIVE}} If the {\tt EXAMPLE\_\-RECURSIVE} tag is set to {\tt YES} then subdirectories will be searched for input files to be used with the $\backslash$include or $\backslash$dontinclude commands irrespective of the value of the {\tt RECURSIVE} tag. Possible values are {\tt YES} and {\tt NO}. If left blank {\tt NO} is used.

\label{config_cfg_example_patterns}
\hypertarget{config_cfg_example_patterns}{}
 \item[{\tt EXAMPLE\_\-PATTERNS} ]\index{EXAMPLE_PATTERNS@{EXAMPLE\_\-PATTERNS}} If the value of the {\tt EXAMPLE\_\-PATH} tag contains directories, you can use the {\tt EXAMPLE\_\-PATTERNS} tag to specify one or more wildcard pattern (like $\ast$.cpp and $\ast$.h) to filter out the source-files in the directories. If left blank all files are included.

\label{config_cfg_image_path}
\hypertarget{config_cfg_image_path}{}
 \item[{\tt IMAGE\_\-PATH} ]\index{IMAGE_PATH@{IMAGE\_\-PATH}} The {\tt IMAGE\_\-PATH} tag can be used to specify one or more files or directories that contain images that are to be included in the documentation (see the \hyperlink{commands_cmdimage}{$\backslash$image} command).

\label{config_cfg_input_filter}
\hypertarget{config_cfg_input_filter}{}
 \item[{\tt INPUT\_\-FILTER} ]\index{INPUT_FILTER@{INPUT\_\-FILTER}} The {\tt INPUT\_\-FILTER} tag can be used to specify a program that doxygen should invoke to filter for each input file. Doxygen will invoke the filter program by executing (via popen()) the command: 

\footnotesize\begin{verbatim}<filter> <input-file>
\end{verbatim}
\normalsize


where $<$filter$>$ is the value of the {\tt INPUT\_\-FILTER} tag, and $<$input-file$>$ is the name of an input file. Doxygen will then use the output that the filter program writes to standard output.

\label{config_cfg_filter_patterns}
\hypertarget{config_cfg_filter_patterns}{}
 \item[{\tt FILTER\_\-PATTERNS} ]\index{FILTER_PATTERNS@{FILTER\_\-PATTERNS}} The {\tt FILTER\_\-PATTERNS} tag can be used to specify filters on a per file pattern basis. Doxygen will compare the file name with each pattern and apply the filter if there is a match. The filters are a list of the form: pattern=filter (like {\tt $\ast$.cpp=my\_\-cpp\_\-filter}). See {\tt INPUT\_\-FILTER} for further info on how filters are used. If {\tt FILTER\_\-PATTERNS} is empty, {\tt INPUT\_\-FILTER} is applied to all files.

\label{config_cfg_filter_source_files}
\hypertarget{config_cfg_filter_source_files}{}
 \item[{\tt FILTER\_\-SOURCE\_\-FILES} ]\index{FILTER_SOURCE_FILES@{FILTER\_\-SOURCE\_\-FILES}} If the {\tt FILTER\_\-SOURCE\_\-FILES} tag is set to {\tt YES}, the input filter (if set using \hyperlink{config_cfg_input_filter}{INPUT\_\-FILTER} ) will also be used to filter the input files that are used for producing the source files to browse (i.e. when SOURCE\_\-BROWSER is set to YES).

\end{description}
\hypertarget{config_sourcebrowser_index}{}\section{Source browsing related options}\label{config_sourcebrowser_index}
\label{config_cfg_source_browser}
\hypertarget{config_cfg_source_browser}{}
 \begin{description}
\item[{\tt SOURCE\_\-BROWSER} ]\index{SOURCE_BROWSER@{SOURCE\_\-BROWSER}} If the {\tt SOURCE\_\-BROWSER} tag is set to {\tt YES} then a list of source files will \par
\char`\"{} be generated. Documented entities will be cross-referenced with these sources. \par
\char`\"{} Note: To get rid of all source code in the generated output, make sure also \par
\char`\"{} {\tt VERBATIM\_\-HEADERS} is set to NO. \par
\char`\"{}

\label{config_cfg_inline_sources}
\hypertarget{config_cfg_inline_sources}{}
 \item[{\tt INLINE\_\-SOURCES} ]\index{INLINE_SOURCES@{INLINE\_\-SOURCES}} Setting the {\tt INLINE\_\-SOURCES} tag to {\tt YES} will include the body of functions, classes and enums directly into the documentation.

\label{config_cfg_strip_code_comments}
\hypertarget{config_cfg_strip_code_comments}{}
 \item[{\tt STRIP\_\-CODE\_\-COMMENTS} ]\index{STRIP_CODE_COMMENTS@{STRIP\_\-CODE\_\-COMMENTS}} Setting the {\tt STRIP\_\-CODE\_\-COMMENTS} tag to {\tt YES} (the default) will instruct doxygen to hide any special comment blocks from generated source code fragments. Normal C and C++ comments will always remain visible.

\label{config_cfg_referenced_by_relation}
\hypertarget{config_cfg_referenced_by_relation}{}
 \item[{\tt REFERENCED\_\-BY\_\-RELATION} ]\index{REFERENCED_BY_RELATION@{REFERENCED\_\-BY\_\-RELATION}} If the {\tt REFERENCED\_\-BY\_\-RELATION} tag is set to {\tt YES} (the default) then for each documented function all documented functions referencing it will be listed.

\label{config_cfg_references_relation}
\hypertarget{config_cfg_references_relation}{}
 \item[{\tt REFERENCES\_\-RELATION} ]\index{REFERENCES_RELATION@{REFERENCES\_\-RELATION}} If the {\tt REFERENCES\_\-RELATION} tag is set to {\tt YES} (the default) then for each documented function all documented entities called/used by that function will be listed.

\label{config_cfg_references_link_source}
\hypertarget{config_cfg_references_link_source}{}
 \item[{\tt REFERENCES\_\-LINK\_\-SOURCE} ]\index{REFERENCES_LINK_SOURCE@{REFERENCES\_\-LINK\_\-SOURCE}} If the {\tt REFERENCES\_\-LINK\_\-SOURCE} tag is set to {\tt YES} (the default) and SOURCE\_\-BROWSER tag is set to {\tt YES}, then the hyperlinks from functions in REFERENCES\_\-RELATION and REFERENCED\_\-BY\_\-RELATION lists will link to the source code. Otherwise they will link to the documentstion.

\label{config_cfg_verbatim_headers}
\hypertarget{config_cfg_verbatim_headers}{}
 \item[{\tt VERBATIM\_\-HEADERS} ]\index{VERBATIM_HEADERS@{VERBATIM\_\-HEADERS}} If the {\tt VERBATIM\_\-HEADERS} tag is set the {\tt YES} (the default) then doxygen will generate a verbatim copy of the header file for each class for which an include is specified. Set to NO to disable this. \begin{Desc}
\item[See also:]Section \hyperlink{commands_cmdclass}{$\backslash$class}.\end{Desc}
\label{config_cfg_use_htags}
\hypertarget{config_cfg_use_htags}{}
 \item[{\tt USE\_\-HTAGS} ]\index{USE_HTAGS@{USE\_\-HTAGS}} If the {\tt USE\_\-HTAGS} tag is set to {\tt YES} then the references to source code will point to the HTML generated by the htags(1) tool instead of doxygen built-in source browser. The htags tool is part of GNU's global source tagging system (see \href{http://www.gnu.org/software/global/global.html}{\tt http://www.gnu.org/software/global/global.html}). The use it do the following:

\begin{enumerate}
\item Install the latest version of global (i.e. 4.8.6 or better)\item Enable SOURCE\_\-BROWSER and USE\_\-HTAGS in the config file\item Make sure the INPUT points to the root of the source tree\item Run doxygen as normal\end{enumerate}


Doxygen will invoke htags (and that will in turn invoke gtags), so these tools must be available from the command line (i.e. in the search path).

The result: instead of the source browser generated by doxygen, the links to source code will now point to the output of htags.

\end{description}
\hypertarget{config_alphabetical_index}{}\section{Alphabetical index options}\label{config_alphabetical_index}
\label{config_cfg_alphabetical_index}
\hypertarget{config_cfg_alphabetical_index}{}
 \begin{description}
\item[{\tt ALPHABETICAL\_\-INDEX} ]\index{ALPHABETICAL_INDEX@{ALPHABETICAL\_\-INDEX}} If the {\tt ALPHABETICAL\_\-INDEX} tag is set to {\tt YES}, an alphabetical index of all compounds will be generated. Enable this if the project contains a lot of classes, structs, unions or interfaces.

\label{config_cfg_cols_in_alpha_index}
\hypertarget{config_cfg_cols_in_alpha_index}{}
 \item[{\tt COLS\_\-IN\_\-ALPHA\_\-INDEX} ]\index{COLS_IN_ALPHA_INDEX@{COLS\_\-IN\_\-ALPHA\_\-INDEX}} If the alphabetical index is enabled (see {\tt ALPHABETICAL\_\-INDEX}) then the {\tt COLS\_\-IN\_\-ALPHA\_\-INDEX} tag can be used to specify the number of columns in which this list will be split (can be a number in the range \mbox{[}1..20\mbox{]})

\label{config_cfg_ignore_prefix}
\hypertarget{config_cfg_ignore_prefix}{}
 \item[{\tt IGNORE\_\-PREFIX} ]\index{IGNORE_PREFIX@{IGNORE\_\-PREFIX}} In case all classes in a project start with a common prefix, all classes will be put under the same header in the alphabetical index. The {\tt IGNORE\_\-PREFIX} tag can be used to specify a prefix (or a list of prefixes) that should be ignored while generating the index headers.

\end{description}
\hypertarget{config_html_output}{}\section{HTML related options}\label{config_html_output}
\label{config_cfg_generate_html}
\hypertarget{config_cfg_generate_html}{}
 \begin{description}
\item[{\tt GENERATE\_\-HTML} ]\index{GENERATE_HTML@{GENERATE\_\-HTML}} If the {\tt GENERATE\_\-HTML} tag is set to {\tt YES} (the default) doxygen will generate HTML output

\label{config_cfg_html_output}
\hypertarget{config_cfg_html_output}{}
 \item[{\tt HTML\_\-OUTPUT} ]\index{HTML_OUTPUT@{HTML\_\-OUTPUT}} The {\tt HTML\_\-OUTPUT} tag is used to specify where the HTML docs will be put. If a relative path is entered the value of {\tt OUTPUT\_\-DIRECTORY} will be put in front of it. If left blank `html' will be used as the default path.

\label{config_cfg_html_file_extension}
\hypertarget{config_cfg_html_file_extension}{}
 \item[{\tt HTML\_\-FILE\_\-EXTENSION} ]\index{HTML_FILE_EXTENSION@{HTML\_\-FILE\_\-EXTENSION}} The {\tt HTML\_\-FILE\_\-EXTENSION} tag can be used to specify the file extension for each generated HTML page (for example: .htm, .php, .asp). If it is left blank doxygen will generate files with .html extension.

\label{config_cfg_html_header}
\hypertarget{config_cfg_html_header}{}
 \item[{\tt HTML\_\-HEADER} ]\index{HTML_HEADER@{HTML\_\-HEADER}} The {\tt HTML\_\-HEADER} tag can be used to specify a user-defined HTML header file for each generated HTML page. To get valid HTML the header file should contain at least a {\tt $<$HTML$>$} and a {\tt $<$BODY$>$} tag, but it is good idea to include the style sheet that is generated by doxygen as well. Minimal example: 

\footnotesize\begin{verbatim}
  <HTML>
    <HEAD>
      <TITLE>My title</TITLE>
      <LINK HREF="doxygen.css" REL="stylesheet" TYPE="text/css">
    </HEAD>
    <BODY BGCOLOR="#FFFFFF">
\end{verbatim}
\normalsize
 If the tag is left blank doxygen will generate a standard header.

The following commands have a special meaning inside the header: {\tt \$title}, {\tt \$datetime}, {\tt \$date}, {\tt \$doxygenversion}, {\tt \$projectname}, and {\tt \$projectnumber}. Doxygen will replace them by respectively the title of the page, the current date and time, only the current date, the version number of doxygen, the project name (see {\tt PROJECT\_\-NAME}), or the project number (see {\tt PROJECT\_\-NUMBER}).

If {\tt CREATE\_\-SUBDIRS} is enabled, the command {\tt \$relpath\$} can be used to produce a relative path to the root of the HTML output directory, e.g. use \$relpath\$doxygen.css, to refer to the standard style sheet.

See also section \hyperlink{doxygen_usage}{Doxygen usage} for information on how to generate the default header that doxygen normally uses.

\label{config_cfg_html_footer}
\hypertarget{config_cfg_html_footer}{}
 \item[{\tt HTML\_\-FOOTER} ]\index{HTML_FOOTER@{HTML\_\-FOOTER}} The {\tt HTML\_\-FOOTER} tag can be used to specify a user-defined HTML footer for each generated HTML page. To get valid HTML the footer file should contain at least a {\tt $<$/BODY$>$} and a {\tt $<$/HTML$>$} tag. A minimal example: 

\footnotesize\begin{verbatim}
    </BODY>
  </HTML>
\end{verbatim}
\normalsize
 If the tag is left blank doxygen will generate a standard footer.

The following commands have a special meaning inside the footer: {\tt \$title}, {\tt \$datetime}, {\tt \$date}, {\tt \$doxygenversion}, {\tt \$projectname}, {\tt \$projectnumber}. Doxygen will replace them by respectively the title of the page, the current date and time, only the current date, the version number of doxygen, the project name (see {\tt PROJECT\_\-NAME}), or the project number (see {\tt PROJECT\_\-NUMBER}).

See also section \hyperlink{doxygen_usage}{Doxygen usage} for information on how to generate the default footer that doxygen normally uses.

\label{config_cfg_html_stylesheet}
\hypertarget{config_cfg_html_stylesheet}{}
 \item[{\tt HTML\_\-STYLESHEET} ]\index{HTML_STYLESHEET@{HTML\_\-STYLESHEET}} The {\tt HTML\_\-STYLESHEET} tag can be used to specify a user-defined cascading style sheet that is used by each HTML page. It can be used to fine-tune the look of the HTML output. If the tag is left blank doxygen will generate a default style sheet.

See also section \hyperlink{doxygen_usage}{Doxygen usage} for information on how to generate the style sheet that doxygen normally uses.

\label{config_cfg_html_align_members}
\hypertarget{config_cfg_html_align_members}{}
 \item[{\tt HTML\_\-ALIGN\_\-MEMBERS} ]\index{HTML_ALIGN_MEMBERS@{HTML\_\-ALIGN\_\-MEMBERS}} If the {\tt HTML\_\-ALIGN\_\-MEMBERS} tag is set to {\tt YES}, the members of classes, files or namespaces will be aligned in HTML using tables. If set to {\tt NO} a bullet list will be used.

{\bf Note:} Setting this tag to {\tt NO} will become obsolete in the future, since I only intent to support and test the aligned representation.

\label{config_cfg_generate_htmlhelp}
\hypertarget{config_cfg_generate_htmlhelp}{}
 \item[{\tt GENERATE\_\-HTMLHELP} ]\index{GENERATE_HTMLHELP@{GENERATE\_\-HTMLHELP}} If the {\tt GENERATE\_\-HTMLHELP} tag is set to {\tt YES} then doxygen generates three additional HTML index files: {\tt index.hhp}, {\tt index.hhc}, and {\tt index.hhk}. The {\tt index.hhp} is a project file that can be read by \href{http://msdn.microsoft.com/library/default.asp?url=/library/en-us/htmlhelp/html/vsconHH1Start.asp}{\tt Microsoft's HTML Help Workshop} on Windows.

The HTML Help Workshop contains a compiler that can convert all HTML output generated by doxygen into a single compiled HTML file (.chm). Compiled HTML files are now used as the Windows 98 help format, and will replace the old Windows help format (.hlp) on all Windows platforms in the future. Compressed HTML files also contain an index, a table of contents, and you can search for words in the documentation. The HTML workshop also contains a viewer for compressed HTML files.

\label{config_cfg_generate_docset}
\hypertarget{config_cfg_generate_docset}{}
 \item[{\tt GENERATE\_\-DOCSET} ]\index{GENERATE_DOCSET@{GENERATE\_\-DOCSET}} If the {\tt GENERATE\_\-DOCSET} tag is set to {\tt YES}, additional index files will be generated that can be used as input for \href{http://developer.apple.com/tools/xcode/}{\tt Apple's Xcode 3 integrated development environment}, introduced with OSX 10.5 (Leopard). To create a documentation set, doxygen will generate a Makefile in the HTML output directory. Running {\tt make} will produce the docset in that directory and running {\tt make install} will install the docset in {\tt $\sim$/Library/Developer/Shared/Documentation/DocSets} so that Xcode will find it at startup.

\label{config_cfg_docset_feedname}
\hypertarget{config_cfg_docset_feedname}{}
 \item[{\tt DOCSET\_\-FEEDNAME} ]\index{DOCSET_FEEDNAME@{DOCSET\_\-FEEDNAME}} When {\tt GENERATE\_\-DOCSET} tag is set to {\tt YES}, this tag determines the name of the feed. A documentation feed provides an umbrella under which multiple documentation sets from a single provider (such as a company or product suite) can be grouped.

\label{config_cfg_docset_bundle_id}
\hypertarget{config_cfg_docset_bundle_id}{}
 \item[{\tt DOCSET\_\-BUNDLE\_\-ID} ]When {\tt GENERATE\_\-DOCSET} tag is set to {\tt YES}, this tag specifies a string that should uniquely identify the documentation set bundle. This should be a reverse domain-name style string, e.g. {\tt com.mycompany.MyDocSet}. Doxygen will append {\tt .docset} to the name.

\label{config_cfg_html_dynamic_sections}
\hypertarget{config_cfg_html_dynamic_sections}{}
 \item[{\tt HTML\_\-DYNAMIC\_\-SECTIONS} ]\index{HTML_DYNAMIC_SECTIONS@{HTML\_\-DYNAMIC\_\-SECTIONS}} If the {\tt HTML\_\-DYNAMIC\_\-SECTIONS} tag is set to {\tt YES} then the generated HTML documentation will contain sections that can be hidden and shown after the page has loaded. For this to work a browser that supports JavaScript and DHTML is required (for instance Mozilla 1.0+, Firefox Netscape 6.0+, Internet explorer 5.0+, Konqueror, or Safari).

\label{config_cfg_chm_file}
\hypertarget{config_cfg_chm_file}{}
 \item[{\tt CHM\_\-FILE} ]\index{CHM_FILE@{CHM\_\-FILE}} If the {\tt GENERATE\_\-HTMLHELP} tag is set to {\tt YES}, the {\tt CHM\_\-FILE} tag can be used to specify the file name of the resulting .chm file. You can add a path in front of the file if the result should not be written to the html output directory.

\label{config_cfg_hhc_location}
\hypertarget{config_cfg_hhc_location}{}
 \item[{\tt HHC\_\-LOCATION} ]\index{HHC_LOCATION@{HHC\_\-LOCATION}} If the {\tt GENERATE\_\-HTMLHELP} tag is set to {\tt YES}, the {\tt HHC\_\-LOCATION} tag can be used to specify the location (absolute path including file name) of the HTML help compiler (hhc.exe). If non empty doxygen will try to run the HTML help compiler on the generated index.hhp.

\label{config_cfg_generate_chi}
\hypertarget{config_cfg_generate_chi}{}
 \item[{\tt GENERATE\_\-CHI} ]\index{GENERATE_CHI@{GENERATE\_\-CHI}} If the {\tt GENERATE\_\-HTMLHELP} tag is set to {\tt YES}, the {\tt GENERATE\_\-CHI} flag controls if a separate .chi index file is generated ({\tt YES}) or that it should be included in the master .chm file ({\tt NO}).

\label{config_cfg_binary_toc}
\hypertarget{config_cfg_binary_toc}{}
 \item[{\tt BINARY\_\-TOC} ]\index{BINARY_TOC@{BINARY\_\-TOC}} If the {\tt GENERATE\_\-HTMLHELP} tag is set to {\tt YES}, the {\tt BINARY\_\-TOC} flag controls whether a binary table of contents is generated ({\tt YES}) or a normal table of contents ({\tt NO}) in the .chm file.

\label{config_cfg_toc_expand}
\hypertarget{config_cfg_toc_expand}{}
 \item[{\tt TOC\_\-EXPAND} ]\index{TOC_EXPAND@{TOC\_\-EXPAND}} The {\tt TOC\_\-EXPAND} flag can be set to YES to add extra items for group members to the table of contents of the HTML help documentation and to the tree view.

\label{config_cfg_disable_index}
\hypertarget{config_cfg_disable_index}{}
 \item[{\tt DISABLE\_\-INDEX} ]\index{DISABLE_INDEX@{DISABLE\_\-INDEX}} If you want full control over the layout of the generated HTML pages it might be necessary to disable the index and replace it with your own. The {\tt DISABLE\_\-INDEX} tag can be used to turn on/off the condensed index at top of each page. A value of NO (the default) enables the index and the value YES disables it.

\label{config_cfg_enum_values_per_line}
\hypertarget{config_cfg_enum_values_per_line}{}
 \item[{\tt ENUM\_\-VALUES\_\-PER\_\-LINE} ]\index{ENUM_VALUES_PER_LINE@{ENUM\_\-VALUES\_\-PER\_\-LINE}} This tag can be used to set the number of enum values (range \mbox{[}1..20\mbox{]}) that doxygen will group on one line in the generated HTML documentation.

\label{config_cfg_generate_treeview}
\hypertarget{config_cfg_generate_treeview}{}
 \item[{\tt GENERATE\_\-TREEVIEW} ]\index{GENERATE_TREEVIEW@{GENERATE\_\-TREEVIEW}} If the {\tt GENERATE\_\-TREEVIEW} tag is set to YES, a side panel will be generated containing a tree-like index structure (just like the one that is generated for HTML Help). For this to work a browser that supports JavaScript and frames is required (for instance Mozilla 1.0+, Netscape 6.0+ or Internet explorer 5.0+ or Konqueror).

\label{config_cfg_treeview_width}
\hypertarget{config_cfg_treeview_width}{}
 \item[{\tt TREEVIEW\_\-WIDTH} ]\index{TREEVIEW_WIDTH@{TREEVIEW\_\-WIDTH}} If the treeview is enabled (see {\tt GENERATE\_\-TREEVIEW}) then this tag can be used to set the initial width (in pixels) of the frame in which the tree is shown.

\end{description}
\hypertarget{config_latex_output}{}\section{LaTeX related options}\label{config_latex_output}
\label{config_cfg_generate_latex}
\hypertarget{config_cfg_generate_latex}{}
 \begin{description}
\item[{\tt GENERATE\_\-LATEX} ]\index{GENERATE_LATEX@{GENERATE\_\-LATEX}} If the {\tt GENERATE\_\-LATEX} tag is set to {\tt YES} (the default) doxygen will generate $\mbox{\LaTeX}$ output.

\label{config_cfg_latex_output}
\hypertarget{config_cfg_latex_output}{}
 \item[{\tt LATEX\_\-OUTPUT} ]\index{LATEX_OUTPUT@{LATEX\_\-OUTPUT}} The {\tt LATEX\_\-OUTPUT} tag is used to specify where the $\mbox{\LaTeX}$ docs will be put. If a relative path is entered the value of {\tt OUTPUT\_\-DIRECTORY} will be put in front of it. If left blank `latex' will be used as the default path.

\label{config_cfg_latex_cmd_name}
\hypertarget{config_cfg_latex_cmd_name}{}
 \item[{\tt LATEX\_\-CMD\_\-NAME} ]\index{LATEX_CMD_NAME@{LATEX\_\-CMD\_\-NAME}} The {\tt LATEX\_\-CMD\_\-NAME} tag can be used to specify the LaTeX command name to be invoked. If left blank `latex' will be used as the default command name.

\label{config_cfg_makeindex_cmd_name}
\hypertarget{config_cfg_makeindex_cmd_name}{}
 \item[{\tt MAKEINDEX\_\-CMD\_\-NAME} ]\index{MAKEINDEX_CMD_NAME@{MAKEINDEX\_\-CMD\_\-NAME}} The MAKEINDEX\_\-CMD\_\-NAME tag can be used to specify the command name to generate index for LaTeX. If left blank `makeindex' will be used as the default command name.

\label{config_cfg_compact_latex}
\hypertarget{config_cfg_compact_latex}{}
 \item[{\tt COMPACT\_\-LATEX} ]\index{COMPACT_LATEX@{COMPACT\_\-LATEX}} If the {\tt COMPACT\_\-LATEX} tag is set to {\tt YES} doxygen generates more compact $\mbox{\LaTeX}$ documents. This may be useful for small projects and may help to save some trees in general.

\label{config_cfg_paper_type}
\hypertarget{config_cfg_paper_type}{}
 \item[{\tt PAPER\_\-TYPE} ]\index{PAPER_TYPE@{PAPER\_\-TYPE}} The {\tt PAPER\_\-TYPE} tag can be used to set the paper type that is used by the printer. Possible values are: \begin{itemize}
\item {\tt a4} (210 x 297 mm). \item {\tt a4wide} (same as a4, but including the a4wide package). \item {\tt letter} (8.5 x 11 inches). \item {\tt legal} (8.5 x 14 inches). \item {\tt executive} (7.25 x 10.5 inches) \end{itemize}
If left blank a4wide will be used.

\label{config_cfg_extra_packages}
\hypertarget{config_cfg_extra_packages}{}
 \item[{\tt EXTRA\_\-PACKAGES} ]\index{EXTRA_PACKAGES@{EXTRA\_\-PACKAGES}} The {\tt EXTRA\_\-PACKAGES} tag can be used to specify one or more $\mbox{\LaTeX}$ package names that should be included in the $\mbox{\LaTeX}$ output. To get the times font for instance you can specify 

\footnotesize\begin{verbatim}
EXTRA_PACKAGES = times
\end{verbatim}
\normalsize
 If left blank no extra packages will be included.

\label{config_cfg_latex_header}
\hypertarget{config_cfg_latex_header}{}
 \item[{\tt LATEX\_\-HEADER} ]\index{LATEX_HEADER@{LATEX\_\-HEADER}} The {\tt LATEX\_\-HEADER} tag can be used to specify a personal $\mbox{\LaTeX}$ header for the generated $\mbox{\LaTeX}$ document. The header should contain everything until the first chapter.

If it is left blank doxygen will generate a standard header. See section \hyperlink{doxygen_usage}{Doxygen usage} for information on how to let doxygen write the default header to a separate file.

\begin{Desc}
\item[Note: ]Only use a user-defined header if you know what you are doing!\end{Desc}
The following commands have a special meaning inside the header: {\tt \$title}, {\tt \$datetime}, {\tt \$date}, {\tt \$doxygenversion}, {\tt \$projectname}, {\tt \$projectnumber}. Doxygen will replace them by respectively the title of the page, the current date and time, only the current date, the version number of doxygen, the project name (see {\tt PROJECT\_\-NAME}), or the project number (see {\tt PROJECT\_\-NUMBER}).

\label{config_cfg_pdf_hyperlinks}
\hypertarget{config_cfg_pdf_hyperlinks}{}
 \item[{\tt PDF\_\-HYPERLINKS} ]\index{PDF_HYPERLINKS@{PDF\_\-HYPERLINKS}}

If the {\tt PDF\_\-HYPERLINKS} tag is set to {\tt YES}, the $\mbox{\LaTeX}$ that is generated is prepared for conversion to PDF (using ps2pdf or pdflatex). The PDF file will contain links (just like the HTML output) instead of page references. This makes the output suitable for online browsing using a PDF viewer.

\label{config_cfg_use_pdflatex}
\hypertarget{config_cfg_use_pdflatex}{}
 \item[{\tt USE\_\-PDFLATEX} ]\index{LATEX_PDFLATEX@{LATEX\_\-PDFLATEX}}

If the {\tt LATEX\_\-PDFLATEX} tag is set to {\tt YES}, doxygen will use pdflatex to generate the PDF file directly from the $\mbox{\LaTeX}$ files.

\label{config_cfg_latex_batchmode}
\hypertarget{config_cfg_latex_batchmode}{}
 \item[{\tt LATEX\_\-BATCHMODE} ]\index{LATEX_BATCHMODE@{LATEX\_\-BATCHMODE}}

If the {\tt LATEX\_\-BATCHMODE} tag is set to {\tt YES}, doxygen will add the $\backslash$batchmode. command to the generated $\mbox{\LaTeX}$ files. This will instruct $\mbox{\LaTeX}$ to keep running if errors occur, instead of asking the user for help. This option is also used when generating formulas in HTML.

\label{config_cfg_latex_hide_indices}
\hypertarget{config_cfg_latex_hide_indices}{}
 \item[{\tt LATEX\_\-HIDE\_\-INDICES} ]\index{LATEX_HIDE_INDICES@{LATEX\_\-HIDE\_\-INDICES}}

If {\tt LATEX\_\-HIDE\_\-INDICES} is set to {\tt YES} then doxygen will not include the index chapters (such as File Index, Compound Index, etc.) in the output.

\end{description}
\hypertarget{config_rtf_output}{}\section{RTF related options}\label{config_rtf_output}
\label{config_cfg_generate_rtf}
\hypertarget{config_cfg_generate_rtf}{}
 \begin{description}
\item[{\tt GENERATE\_\-RTF} ]\index{GENERATE_RTF@{GENERATE\_\-RTF}} If the {\tt GENERATE\_\-RTF} tag is set to {\tt YES} doxygen will generate RTF output. The RTF output is optimized for Word 97 and may not look too pretty with other readers/editors.

\label{config_cfg_rtf_output}
\hypertarget{config_cfg_rtf_output}{}
 \item[{\tt RTF\_\-OUTPUT} ]\index{RTF_OUTPUT@{RTF\_\-OUTPUT}} The {\tt RTF\_\-OUTPUT} tag is used to specify where the RTF docs will be put. If a relative path is entered the value of {\tt OUTPUT\_\-DIRECTORY} will be put in front of it. If left blank {\tt rtf} will be used as the default path.

\label{config_cfg_compact_rtf}
\hypertarget{config_cfg_compact_rtf}{}
 \item[{\tt COMPACT\_\-RTF} ]\index{COMPACT_RTF@{COMPACT\_\-RTF}} If the {\tt COMPACT\_\-RTF} tag is set to {\tt YES} doxygen generates more compact RTF documents. This may be useful for small projects and may help to save some trees in general.

\label{config_cfg_rtf_hyperlinks}
\hypertarget{config_cfg_rtf_hyperlinks}{}
 \item[{\tt RTF\_\-HYPERLINKS} ]\index{RTF_HYPERLINKS@{RTF\_\-HYPERLINKS}} If the {\tt RTF\_\-HYPERLINKS} tag is set to {\tt YES}, the RTF that is generated will contain hyperlink fields. The RTF file will contain links (just like the HTML output) instead of page references. This makes the output suitable for online browsing using Word or some other Word compatible reader that support those fields.

\begin{Desc}
\item[note:]WordPad (write) and others do not support links.\end{Desc}
\label{config_cfg_rtf_stylesheet_file}
\hypertarget{config_cfg_rtf_stylesheet_file}{}
 \item[{\tt RTF\_\-STYLESHEET\_\-FILE} ]\index{RTF_STYLESHEET_FILE@{RTF\_\-STYLESHEET\_\-FILE}} Load stylesheet definitions from file. Syntax is similar to doxygen's config file, i.e. a series of assignments. You only have to provide replacements, missing definitions are set to their default value.

See also section \hyperlink{doxygen_usage}{Doxygen usage} for information on how to generate the default style sheet that doxygen normally uses.

\label{config_cfg_rtf_extensions_file}
\hypertarget{config_cfg_rtf_extensions_file}{}
 \item[{\tt RTF\_\-EXTENSIONS\_\-FILE} ]Set optional variables used in the generation of an RTF document. Syntax is similar to doxygen's config file. A template extensions file can be generated using {\tt doxygen -e rtf extensionFile}.

\end{description}
\hypertarget{config_man_output}{}\section{Man page related options}\label{config_man_output}
\label{config_cfg_generate_man}
\hypertarget{config_cfg_generate_man}{}
 \begin{description}
\item[{\tt GENERATE\_\-MAN} ]\index{GENERATE_MAN@{GENERATE\_\-MAN}} If the {\tt GENERATE\_\-MAN} tag is set to {\tt YES} (the default) doxygen will generate man pages for classes and files.

\label{config_cfg_man_output}
\hypertarget{config_cfg_man_output}{}
 \item[{\tt MAN\_\-OUTPUT} ]\index{MAN_OUTPUT@{MAN\_\-OUTPUT}} The {\tt MAN\_\-OUTPUT} tag is used to specify where the man pages will be put. If a relative path is entered the value of {\tt OUTPUT\_\-DIRECTORY} will be put in front of it. If left blank `man' will be used as the default path. A directory man3 will be created inside the directory specified by {\tt MAN\_\-OUTPUT}.

\label{config_cfg_man_extension}
\hypertarget{config_cfg_man_extension}{}
 \item[{\tt MAN\_\-EXTENSION} ]\index{MAX_EXTENSION@{MAX\_\-EXTENSION}} The {\tt MAN\_\-EXTENSION} tag determines the extension that is added to the generated man pages (default is the subroutine's section .3)

\label{config_cfg_man_links}
\hypertarget{config_cfg_man_links}{}
 \item[{\tt MAN\_\-LINKS} ]\index{MAN_LINKS@{MAN\_\-LINKS}} If the {\tt MAN\_\-LINKS} tag is set to {\tt YES} and doxygen generates man output, then it will generate one additional man file for each entity documented in the real man page(s). These additional files only source the real man page, but without them the man command would be unable to find the correct page. The default is {\tt NO}.

\end{description}
\hypertarget{config_xml_output}{}\section{XML related options}\label{config_xml_output}
\label{config_cfg_generate_xml}
\hypertarget{config_cfg_generate_xml}{}
 \begin{description}
\item[{\tt GENERATE\_\-XML} ]\index{GENERATE_XML@{GENERATE\_\-XML}} If the {\tt GENERATE\_\-XML} tag is set to {\tt YES} Doxygen will generate an XML file that captures the structure of the code including all documentation.

\label{config_cfg_xml_output}
\hypertarget{config_cfg_xml_output}{}
 \item[{\tt XML\_\-OUTPUT} ]\index{XML_OUTPUT@{XML\_\-OUTPUT}} The {\tt XML\_\-OUTPUT} tag is used to specify where the XML pages will be put. If a relative path is entered the value of {\tt OUTPUT\_\-DIRECTORY} will be put in front of it. If left blank {\tt xml} will be used as the default path.

\label{config_cfg_xml_schema}
\hypertarget{config_cfg_xml_schema}{}
 \item[{\tt XML\_\-SCHEMA} ]\index{XML_SCHEMA@{XML\_\-SCHEMA}} The {\tt XML\_\-SCHEMA} tag can be used to specify an XML schema, which can be used by a validating XML parser to check the syntax of the XML files.

\label{config_cfg_xml_dtd}
\hypertarget{config_cfg_xml_dtd}{}
 \item[{\tt XML\_\-DTD} ]\index{XML_DTD@{XML\_\-DTD}} The {\tt XML\_\-DTD} tag can be used to specify an XML DTD, which can be used by a validating XML parser to check the syntax of the XML files.

\label{config_cfg_xml_programlisting}
\hypertarget{config_cfg_xml_programlisting}{}
 \item[{\tt XML\_\-PROGRAMLISTING} ]\index{XML_PROGRAMLISTING@{XML\_\-PROGRAMLISTING}} If the {\tt XML\_\-PROGRAMLISTING} tag is set to {\tt YES} Doxygen will dump the program listings (including syntax highlighting and cross-referencing information) to the XML output. Note that enabling this will significantly increase the size of the XML output.

\end{description}
\hypertarget{config_autogen_output}{}\section{AUTOGEN\_\-DEF related options}\label{config_autogen_output}
\label{config_cfg_generate_autogen_def}
\hypertarget{config_cfg_generate_autogen_def}{}
 \begin{description}
\item[{\tt GENERATE\_\-AUTOGEN\_\-DEF} ]\index{GENERATE_AUTOGEN_DEF@{GENERATE\_\-AUTOGEN\_\-DEF}} If the {\tt GENERATE\_\-AUTOGEN\_\-DEF} tag is set to {\tt YES} Doxygen will generate an AutoGen Definitions (see \href{http://autogen.sf.net}{\tt http://autogen.sf.net}) file that captures the structure of the code including all documentation. Note that this feature is still experimental and incomplete at the moment.

\end{description}
\hypertarget{config_perlmod_output}{}\section{PERLMOD related options}\label{config_perlmod_output}
\label{config_cfg_generate_perlmod}
\hypertarget{config_cfg_generate_perlmod}{}
 \begin{description}
\item[{\tt GENERATE\_\-PERLMOD} ]\index{GENERATE_PERLMOD@{GENERATE\_\-PERLMOD}} If the {\tt GENERATE\_\-PERLMOD} tag is set to {\tt YES} Doxygen will generate a Perl module file that captures the structure of the code including all documentation. Note that this feature is still experimental and incomplete at the moment.

\label{config_cfg_perlmod_latex}
\hypertarget{config_cfg_perlmod_latex}{}
 \item[{\tt PERLMOD\_\-LATEX} ]\index{PERLMOD_LATEX@{PERLMOD\_\-LATEX}} If the {\tt PERLMOD\_\-LATEX} tag is set to {\tt YES} Doxygen will generate the necessary Makefile rules, Perl scripts and LaTeX code to be able to generate PDF and DVI output from the Perl module output.

\label{config_cfg_perlmod_pretty}
\hypertarget{config_cfg_perlmod_pretty}{}
 \item[{\tt PERLMOD\_\-PRETTY} ]\index{PERLMOD_PRETTY@{PERLMOD\_\-PRETTY}} If the {\tt PERLMOD\_\-PRETTY} tag is set to {\tt YES} the Perl module output will be nicely formatted so it can be parsed by a human reader. This is useful if you want to understand what is going on. On the other hand, if this tag is set to {\tt NO} the size of the Perl module output will be much smaller and Perl will parse it just the same.

\label{config_cfg_perlmod_makevar_prefix}
\hypertarget{config_cfg_perlmod_makevar_prefix}{}
 \item[{\tt PERLMOD\_\-MAKEVAR\_\-PREFIX} ]\index{PERLMOD_MAKEVAR_PREFIX@{PERLMOD\_\-MAKEVAR\_\-PREFIX}} The names of the make variables in the generated doxyrules.make file are prefixed with the string contained in {\tt PERLMOD\_\-MAKEVAR\_\-PREFIX}. This is useful so different doxyrules.make files included by the same Makefile don't overwrite each other's variables.

\end{description}
\hypertarget{config_config_prepro}{}\section{Preprocessor related options}\label{config_config_prepro}
\label{config_cfg_enable_preprocessing}
\hypertarget{config_cfg_enable_preprocessing}{}
 \begin{description}
\item[{\tt ENABLE\_\-PREPROCESSING} ]\index{ENABLE_PREPROCESSING@{ENABLE\_\-PREPROCESSING}} If the {\tt ENABLE\_\-PREPROCESSING} tag is set to {\tt YES} (the default) doxygen will evaluate all C-preprocessor directives found in the sources and include files.

\label{config_cfg_macro_expansion}
\hypertarget{config_cfg_macro_expansion}{}
 \item[{\tt MACRO\_\-EXPANSION} ]\index{MACRO_EXPANSION@{MACRO\_\-EXPANSION}} If the {\tt MACRO\_\-EXPANSION} tag is set to {\tt YES} doxygen will expand all macro names in the source code. If set to {\tt NO} (the default) only conditional compilation will be performed. Macro expansion can be done in a controlled way by setting {\tt EXPAND\_\-ONLY\_\-PREDEF} to {\tt YES}.

\label{config_cfg_expand_only_predef}
\hypertarget{config_cfg_expand_only_predef}{}
 \item[{\tt EXPAND\_\-ONLY\_\-PREDEF} ]\index{EXPAND_ONLY_PREDEF@{EXPAND\_\-ONLY\_\-PREDEF}} If the {\tt EXPAND\_\-ONLY\_\-PREDEF} and {\tt MACRO\_\-EXPANSION} tags are both set to YES then the macro expansion is limited to the macros specified with the {\tt PREDEFINED} and {\tt EXPAND\_\-AS\_\-DEFINED} tags.

\label{config_cfg_search_includes}
\hypertarget{config_cfg_search_includes}{}
 \item[{\tt SEARCH\_\-INCLUDES} ]\index{SEARCH_INCLUDES @{SEARCH\_\-INCLUDES }} If the {\tt SEARCH\_\-INCLUDES} tag is set to {\tt YES} (the default) the includes files in the {\tt INCLUDE\_\-PATH} (see below) will be searched if a \#include is found.

\label{config_cfg_include_path}
\hypertarget{config_cfg_include_path}{}
 \item[{\tt INCLUDE\_\-PATH} ]\index{INCLUDE_PATH@{INCLUDE\_\-PATH}} The {\tt INCLUDE\_\-PATH} tag can be used to specify one or more directories that contain include files that are not input files but should be processed by the preprocessor.

\label{config_cfg_predefined}
\hypertarget{config_cfg_predefined}{}
 \item[{\tt PREDEFINED} ]\index{PREDEFINED@{PREDEFINED}} The {\tt PREDEFINED} tag can be used to specify one or more macro names that are defined before the preprocessor is started (similar to the -D option of gcc). The argument of the tag is a list of macros of the form: {\tt name} or {\tt name=definition} (no spaces). If the definition and the \char`\"{}=\char`\"{} are omitted, \char`\"{}=1\char`\"{} is assumed. To prevent a macro definition from being undefined via \#undef or recursively expanded use the := operator instead of the = operator.

\label{config_cfg_expand_as_defined}
\hypertarget{config_cfg_expand_as_defined}{}
 \item[{\tt EXPAND\_\-AS\_\-DEFINED} ]\index{EXPAND_AS_DEFINED@{EXPAND\_\-AS\_\-DEFINED}} If the {\tt MACRO\_\-EXPANSION} and {\tt EXPAND\_\-ONLY\_\-PREDEF} tags are set to {\tt YES} then this tag can be used to specify a list of macro names that should be expanded. The macro definition that is found in the sources will be used. Use the {\tt PREDEFINED} tag if you want to use a different macro definition.

\label{config_cfg_skip_function_macros}
\hypertarget{config_cfg_skip_function_macros}{}
 \item[{\tt SKIP\_\-FUNCTION\_\-MACROS} ]\index{SKIP_FUNCTION_MACROS@{SKIP\_\-FUNCTION\_\-MACROS}} If the {\tt SKIP\_\-FUNCTION\_\-MACROS} tag is set to {\tt YES} (the default) then doxygen's preprocessor will remove all function-like macros that are alone on a line, have an all uppercase name, and do not end with a semicolon. Such function macros are typically used for boiler-plate code, and will confuse the parser if not removed.

\end{description}
\hypertarget{config_config_extref}{}\section{External reference options}\label{config_config_extref}
\label{config_cfg_tagfiles}
\hypertarget{config_cfg_tagfiles}{}
 \begin{description}
\item[{\tt TAGFILES} ]\index{TAGFILES@{TAGFILES}} The {\tt TAGFILES} tag can be used to specify one or more tagfiles.

See section \hyperlink{doxytag_usage}{Doxytag usage} for more information about the usage of tag files.

Optionally an initial location of the external documentation can be added for each tagfile. The format of a tag file without this location is as follows: \small\begin{alltt}
TAGFILES = file1 file2 ... \end{alltt}
\normalsize 
 Adding location for the tag files is done as follows: \small\begin{alltt}
TAGFILES = file1=loc1 "file2 = loc2" ... \end{alltt}
\normalsize 
 where {\tt loc1} and {\tt loc2} can be relative or absolute paths or URLs, If a location is present for each tag, the installdox tool (see section \hyperlink{installdox_usage}{Installdox usage} for more information) does not have to be run to correct the links.

\begin{Desc}
\item[Note:]Each tag file must have a unique name (where the name does {\em not\/} include the path) If a tag file is not located in the directory in which doxygen is run, you must also specify the path to the tagfile here.\end{Desc}
\label{config_cfg_generate_tagfile}
\hypertarget{config_cfg_generate_tagfile}{}
 \item[{\tt GENERATE\_\-TAGFILE} ]\index{GENERATE_TAGFILE@{GENERATE\_\-TAGFILE}} When a file name is specified after {\tt GENERATE\_\-TAGFILE}, doxygen will create a tag file that is based on the input files it reads. See section \hyperlink{doxytag_usage}{Doxytag usage} for more information about the usage of tag files.

\label{config_cfg_allexternals}
\hypertarget{config_cfg_allexternals}{}
 \item[{\tt ALLEXTERNALS} ]\index{ALLEXTERNALS@{ALLEXTERNALS}} If the {\tt ALLEXTERNALS} tag is set to {\tt YES} all external class will be listed in the class index. If set to {\tt NO} only the inherited external classes will be listed.

\label{config_cfg_external_groups}
\hypertarget{config_cfg_external_groups}{}
 \item[{\tt EXTERNAL\_\-GROUPS} ]\index{EXTERNAL_GROUPS@{EXTERNAL\_\-GROUPS}} If the {\tt EXTERNAL\_\-GROUPS} tag is set to {\tt YES} all external groups will be listed in the modules index. If set to {\tt NO}, only the current project's groups will be listed.

\label{config_cfg_perl_path}
\hypertarget{config_cfg_perl_path}{}
 \item[{\tt PERL\_\-PATH} ]\index{PERL_PATH@{PERL\_\-PATH}} The {\tt PERL\_\-PATH} should be the absolute path and name of the perl script interpreter (i.e. the result of `{\tt which perl}').

\end{description}
\hypertarget{config_config_dot}{}\section{Dot options}\label{config_config_dot}
\label{config_cfg_class_diagrams}
\hypertarget{config_cfg_class_diagrams}{}
 \begin{description}
\item[{\tt CLASS\_\-DIAGRAMS} ]\index{CLASS_DIAGRAMS@{CLASS\_\-DIAGRAMS}} If the {\tt CLASS\_\-DIAGRAMS} tag is set to {\tt YES} (the default) doxygen will generate a class diagram (in HTML and $\mbox{\LaTeX}$) for classes with base or super classes. Setting the tag to {\tt NO} turns the diagrams off. Note that this option is superseded by the HAVE\_\-DOT option below. This is only a fallback. It is recommended to install and use dot, since it yields more powerful graphs.

\label{config_cfg_mscgen_path}
\hypertarget{config_cfg_mscgen_path}{}
 \item[{\tt MSCGEN\_\-PATH} ]\index{MSCGEN_PATH@{MSCGEN\_\-PATH}} You can define message sequence charts within doxygen comments using the \hyperlink{commands_cmdmsc}{$\backslash$msc} command. Doxygen will then run the \href{http://www.mcternan.me.uk/mscgen/}{\tt msgen tool}) to produce the chart and insert it in the documentation. The {\tt MSCGEN\_\-PATH} tag allows you to specify the directory where the mscgen tool resides. If left empty the tool is assumed to be found in the default search path.

\label{config_cfg_have_dot}
\hypertarget{config_cfg_have_dot}{}
 \item[{\tt HAVE\_\-DOT} ]\index{HAVE_DOT@{HAVE\_\-DOT}} If you set the {\tt HAVE\_\-DOT} tag to {\tt YES} then doxygen will assume the dot tool is available from the path. This tool is part of \href{http://www.research.att.com/sw/tools/graphviz/}{\tt Graphviz}, a graph visualization toolkit from AT\&T and Lucent Bell Labs. The other options in this section have no effect if this option is set to {\tt NO} (the default)

\label{config_cfg_class_graph}
\hypertarget{config_cfg_class_graph}{}
 \item[{\tt CLASS\_\-GRAPH} ]\index{CLASS_GRAPH@{CLASS\_\-GRAPH}} If the {\tt CLASS\_\-GRAPH} and {\tt HAVE\_\-DOT} tags are set to {\tt YES} then doxygen will generate a graph for each documented class showing the direct and indirect inheritance relations. Setting this tag to {\tt YES} will force the the {\tt CLASS\_\-DIAGRAMS} tag to NO.

\label{config_cfg_collaboration_graph}
\hypertarget{config_cfg_collaboration_graph}{}
 \item[{\tt COLLABORATION\_\-GRAPH} ]\index{COLLABORATION_GRAPH@{COLLABORATION\_\-GRAPH}} If the {\tt COLLABORATION\_\-GRAPH} and {\tt HAVE\_\-DOT} tags are set to {\tt YES} then doxygen will generate a graph for each documented class showing the direct and indirect implementation dependencies (inheritance, containment, and class references variables) of the class with other documented classes.

\label{config_cfg_group_graphs}
\hypertarget{config_cfg_group_graphs}{}
 \item[{\tt GROUP\_\-GRAPHS} ]\index{GROUP_GRAPHS@{GROUP\_\-GRAPHS}} If the GROUP\_\-GRAPHS and HAVE\_\-DOT tags are set to YES then doxygen will generate a graph for groups, showing the direct groups dependencies.

\label{config_cfg_uml_look}
\hypertarget{config_cfg_uml_look}{}
 \item[{\tt UML\_\-LOOK} ]\index{UML_LOOK@{UML\_\-LOOK}} If the UML\_\-LOOK tag is set to YES doxygen will generate inheritance and collaboration diagrams in a style similar to the OMG's Unified Modeling Language.

\label{config_cfg_template_relations}
\hypertarget{config_cfg_template_relations}{}
 \item[{\tt TEMPLATE\_\-RELATIONS} ]\index{TEMPLATE_RELATIONS@{TEMPLATE\_\-RELATIONS}} If the {\tt TEMPLATE\_\-RELATIONS} and {\tt HAVE\_\-DOT} tags are set to {\tt YES} then doxygen will show the relations between templates and their instances.

\label{config_cfg_hide_undoc_relations}
\hypertarget{config_cfg_hide_undoc_relations}{}
 \item[{\tt HIDE\_\-UNDOC\_\-RELATIONS} ]\index{HIDE_UNDOC_RELATIONS@{HIDE\_\-UNDOC\_\-RELATIONS}} If set to YES, the inheritance and collaboration graphs will hide inheritance and usage relations if the target is undocumented or is not a class.

\label{config_cfg_include_graph}
\hypertarget{config_cfg_include_graph}{}
 \item[{\tt INCLUDE\_\-GRAPH} ]\index{INCLUDE_GRAPH @{INCLUDE\_\-GRAPH }} If the {\tt ENABLE\_\-PREPROCESSING}, {\tt SEARCH\_\-INCLUDES}, {\tt INCLUDE\_\-GRAPH}, and {\tt HAVE\_\-DOT} tags are set to {\tt YES} then doxygen will generate a graph for each documented file showing the direct and indirect include dependencies of the file with other documented files.

\label{config_cfg_included_by_graph}
\hypertarget{config_cfg_included_by_graph}{}
 \item[{\tt INCLUDED\_\-BY\_\-GRAPH} ]\index{INCLUDED_BY_GRAPH @{INCLUDED\_\-BY\_\-GRAPH }} If the {\tt ENABLE\_\-PREPROCESSING}, {\tt SEARCH\_\-INCLUDES}, {\tt INCLUDED\_\-BY\_\-GRAPH}, and {\tt HAVE\_\-DOT} tags are set to {\tt YES} then doxygen will generate a graph for each documented header file showing the documented files that directly or indirectly include this file.

\label{config_cfg_call_graph}
\hypertarget{config_cfg_call_graph}{}
 \item[{\tt CALL\_\-GRAPH} ]\index{CALL_GRAPH@{CALL\_\-GRAPH}} If the {\tt CALL\_\-GRAPH} and {\tt HAVE\_\-DOT} tags are set to {\tt YES} then doxygen will generate a call dependency graph for every global function or class method. Note that enabling this option will significantly increase the time of a run. So in most cases it will be better to enable call graphs for selected functions only using the $\backslash$callgraph command.

\label{config_cfg_caller_graph}
\hypertarget{config_cfg_caller_graph}{}
 \item[{\tt CALLER\_\-GRAPH} ]\index{CALLER_GRAPH@{CALLER\_\-GRAPH}} If the {\tt CALLER\_\-GRAPH} and {\tt HAVE\_\-DOT} tags are set to {\tt YES} then doxygen will generate a caller dependency graph for every global function or class method. Note that enabling this option will significantly increase the time of a run. So in most cases it will be better to enable caller graphs for selected functions only using the $\backslash$callergraph command.

\label{config_cfg_graphical_hierarchy}
\hypertarget{config_cfg_graphical_hierarchy}{}
 \item[{\tt GRAPHICAL\_\-HIERARCHY} ]\index{GRAPHICAL_HIERARCHY@{GRAPHICAL\_\-HIERARCHY}} If the {\tt GRAPHICAL\_\-HIERARCHY} and {\tt HAVE\_\-DOT} tags are set to {\tt YES} then doxygen will graphical hierarchy of all classes instead of a textual one.

\label{config_cfg_directory_graph}
\hypertarget{config_cfg_directory_graph}{}
 \item[{\tt DIRECTORY\_\-GRAPH} ]\index{DIRECTORY_GRAPH @{DIRECTORY\_\-GRAPH }} If the {\tt DIRECTORY\_\-GRAPH}, {\tt SHOW\_\-DIRECTORIES} and {\tt HAVE\_\-DOT} options are set to {\tt YES} then doxygen will show the dependencies a directory has on other directories in a graphical way. The dependency relations are determined by the \#include relations between the files in the directories.

\label{config_cfg_dot_graph_max_nodes}
\hypertarget{config_cfg_dot_graph_max_nodes}{}
 \item[{\tt DOT\_\-GRAPH\_\-MAX\_\-NODES} ]\index{DOT_GRAPH_MAX_NODES@{DOT\_\-GRAPH\_\-MAX\_\-NODES}} The {\tt MAX\_\-DOT\_\-GRAPH\_\-MAX\_\-NODES} tag can be used to set the maximum number of nodes that will be shown in the graph. If the number of nodes in a graph becomes larger than this value, doxygen will truncate the graph, which is visualized by representing a node as a red box. Note that doxygen if the number of direct children of the root node in a graph is already larger than {\tt DOT\_\-GRAPH\_\-MAX\_\-NODES} then the graph will not be shown at all. Also note that the size of a graph can be further restricted by {\tt MAX\_\-DOT\_\-GRAPH\_\-DEPTH}.

\label{config_cfg_max_dot_graph_depth}
\hypertarget{config_cfg_max_dot_graph_depth}{}
 \item[{\tt MAX\_\-DOT\_\-GRAPH\_\-DEPTH} ]\index{MAX_DOT_GRAPH_DEPTH@{MAX\_\-DOT\_\-GRAPH\_\-DEPTH}} The {\tt MAX\_\-DOT\_\-GRAPH\_\-DEPTH} tag can be used to set the maximum depth of the graphs generated by dot. A depth value of 3 means that only nodes reachable from the root by following a path via at most 3 edges will be shown. Nodes that lay further from the root node will be omitted. Note that setting this option to 1 or 2 may greatly reduce the computation time needed for large code bases. Also note that the size of a graph can be further restricted by {\tt DOT\_\-GRAPH\_\-MAX\_\-NODES}. Using a depth of 0 means no depth restriction (the default).

\label{config_cfg_dot_image_format}
\hypertarget{config_cfg_dot_image_format}{}
 \item[{\tt DOT\_\-IMAGE\_\-FORMAT} ]\index{DOT_IMAGE_FORMAT@{DOT\_\-IMAGE\_\-FORMAT}} The {\tt DOT\_\-IMAGE\_\-FORMAT} tag can be used to set the image format of the images generated by dot. Possible values are gif, jpg, and png. If left blank png will be used.

\label{config_cfg_dot_path}
\hypertarget{config_cfg_dot_path}{}
 \item[{\tt DOT\_\-PATH} ]\index{DOT_PATH@{DOT\_\-PATH}} This tag can be used to specify the path where the dot tool can be found. If left blank, it is assumed the dot tool can be found on the path.

\label{config_cfg_dotfile_dirs}
\hypertarget{config_cfg_dotfile_dirs}{}
 \item[{\tt DOTFILE\_\-DIRS} ]\index{DOTFILE_DIRS@{DOTFILE\_\-DIRS}} This tag can be used to specify one or more directories that contain dot files that are included in the documentation (see the $\backslash$dotfile command).

\label{config_cfg_dot_transparent}
\hypertarget{config_cfg_dot_transparent}{}
 \item[{\tt DOT\_\-TRANSPARENT} ]\index{DOT_TRANSPARENT@{DOT\_\-TRANSPARENT}} Set the {\tt DOT\_\-TRANSPARENT} tag to {\tt YES} to generate images with a transparent background. This is enabled by default, which results in a transparent background. Warning: Depending on the platform used, enabling this option may lead to badly anti-aliased labels on the edges of a graph (i.e. they become hard to read).

\label{config_cfg_dot_multi_targets}
\hypertarget{config_cfg_dot_multi_targets}{}
 \item[{\tt DOT\_\-MULTI\_\-TARGETS} ]\index{DOT_MULTI_TARGET@{DOT\_\-MULTI\_\-TARGET}} Set the {\tt DOT\_\-MULTI\_\-TARGETS} tag to {\tt YES} allow dot to generate multiple output files in one run (i.e. multiple -o and -T options on the command line). This makes dot run faster, but since only newer versions of dot ($>$1.8.10) support this, this feature is disabled by default.

\label{config_cfg_generate_legend}
\hypertarget{config_cfg_generate_legend}{}
 \item[{\tt GENERATE\_\-LEGEND} ]\index{GENERATE_LEGEND@{GENERATE\_\-LEGEND}} If the {\tt GENERATE\_\-LEGEND} tag is set to {\tt YES} (the default) doxygen will generate a legend page explaining the meaning of the various boxes and arrows in the dot generated graphs.

\label{config_cfg_dot_cleanup}
\hypertarget{config_cfg_dot_cleanup}{}
 \item[{\tt DOT\_\-CLEANUP} ]\index{DOT_CLEANUP@{DOT\_\-CLEANUP}} If the {\tt DOT\_\-CLEANUP} tag is set to {\tt YES} (the default) doxygen will remove the intermediate dot files that are used to generate the various graphs.

\end{description}
\hypertarget{config_config_search}{}\section{Search engine options}\label{config_config_search}
\label{config_cfg_searchengine}
\hypertarget{config_cfg_searchengine}{}
 \begin{description}
\item[{\tt SEARCHENGINE} ]\index{SEARCHENGINE@{SEARCHENGINE}} The {\tt SEARCHENGINE} tag specifies whether or not the HTML output should contain a search facility. Possible values are {\tt YES} and {\tt NO}. If set to YES, doxygen will produce a search index and a PHP script to search through the index. For this to work the documentation should be viewed via a web-server running PHP version 4.1.0 or higher. (See \href{http://www.php.net/manual/en/installation.php}{\tt http://www.php.net/manual/en/installation.php} for installation instructions).

\end{description}
\subsection*{Examples}

Suppose you have a simple project consisting of two files: a source file {\tt example.cc} and a header file {\tt example.h}. Then a minimal configuration file is as simple as: 

\footnotesize\begin{verbatim}
INPUT            = example.cc example.h
\end{verbatim}
\normalsize


Assuming the example makes use of Qt classes and perl is located in {\tt /usr/bin}, a more realistic configuration file would be: 

\footnotesize\begin{verbatim}
PROJECT_NAME     = Example
INPUT            = example.cc example.h
WARNINGS         = YES
TAGFILES         = qt.tag
PERL_PATH        = /usr/bin/perl
SEARCHENGINE     = NO
\end{verbatim}
\normalsize


To generate the documentation for the \href{http://www.stack.nl/~dimitri/qdbttabular/index.html}{\tt QdbtTabular} package I have used the following configuration file: 

\footnotesize\begin{verbatim}
PROJECT_NAME     = QdbtTabular
OUTPUT_DIRECTORY = html
WARNINGS         = YES
INPUT            = examples/examples.doc src
FILE_PATTERNS    = *.cc *.h
INCLUDE_PATH     = examples
TAGFILES         = qt.tag
PERL_PATH        = /usr/local/bin/perl
SEARCHENGINE     = YES
\end{verbatim}
\normalsize


To regenerate the Qt-1.44 documentation from the sources, you could use the following config file: 

\footnotesize\begin{verbatim}
PROJECT_NAME         = Qt
OUTPUT_DIRECTORY     = qt_docs
HIDE_UNDOC_MEMBERS   = YES
HIDE_UNDOC_CLASSES   = YES
ENABLE_PREPROCESSING = YES
MACRO_EXPANSION      = YES
EXPAND_ONLY_PREDEF   = YES
SEARCH_INCLUDES      = YES
FULL_PATH_NAMES      = YES
STRIP_FROM_PATH      = $(QTDIR)/
PREDEFINED           = USE_TEMPLATECLASS Q_EXPORT= \
                       QArrayT:=QArray \
                       QListT:=QList \
                       QDictT:=QDict \
                       QQueueT:=QQueue \
                       QVectorT:=QVector \
                       QPtrDictT:=QPtrDict \
                       QIntDictT:=QIntDict \
                       QStackT:=QStack \
                       QDictIteratorT:=QDictIterator \
                       QListIteratorT:=QListIterator \
                       QCacheT:=QCache \
                       QCacheIteratorT:=QCacheIterator \
                       QIntCacheT:=QIntCache \
                       QIntCacheIteratorT:=QIntCacheIterator \
                       QIntDictIteratorT:=QIntDictIterator \
                       QPtrDictIteratorT:=QPtrDictIterator
INPUT                = $(QTDIR)/doc \
                       $(QTDIR)/src/widgets \
                       $(QTDIR)/src/kernel \
                       $(QTDIR)/src/dialogs \
                       $(QTDIR)/src/tools
FILE_PATTERNS        = *.cpp *.h q*.doc
INCLUDE_PATH         = $(QTDIR)/include 
RECURSIVE            = YES
\end{verbatim}
\normalsize


For the Qt-2.1 sources I recommend to use the following settings: 

\footnotesize\begin{verbatim}
PROJECT_NAME          = Qt
PROJECT_NUMBER        = 2.1
HIDE_UNDOC_MEMBERS    = YES
HIDE_UNDOC_CLASSES    = YES
SOURCE_BROWSER        = YES
INPUT                 = $(QTDIR)/src
FILE_PATTERNS         = *.cpp *.h q*.doc
RECURSIVE             = YES
EXCLUDE_PATTERNS      = *codec.cpp moc_* */compat/* */3rdparty/*
ALPHABETICAL_INDEX    = YES
COLS_IN_ALPHA_INDEX   = 3
IGNORE_PREFIX         = Q
ENABLE_PREPROCESSING  = YES
MACRO_EXPANSION       = YES
INCLUDE_PATH          = $(QTDIR)/include
PREDEFINED            = Q_PROPERTY(x)= \
                        Q_OVERRIDE(x)= \
                        Q_EXPORT= \
                        Q_ENUMS(x)= \
                        "QT_STATIC_CONST=static const " \
                        _WS_X11_ \
                        INCLUDE_MENUITEM_DEF
EXPAND_ONLY_PREDEF    = YES
EXPAND_AS_DEFINED     = Q_OBJECT_FAKE Q_OBJECT ACTIVATE_SIGNAL_WITH_PARAM \
                        Q_VARIANT_AS
\end{verbatim}
\normalsize


Here doxygen's preprocessor is used to substitute some macro names that are normally substituted by the C preprocessor, but without doing full macro expansion. 