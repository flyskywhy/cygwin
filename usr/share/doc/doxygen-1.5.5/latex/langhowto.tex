\subsubsection*{Support for multiple languages}

Doxygen has built-in support for multiple languages. This means that the text fragments, generated by doxygen, can be produced in languages other than English (the default). The output language is chosen through the configuration file (with default name and known as Doxyfile).

Currently (version 1.5.5), 34 languages are supported (sorted alphabetically): Afrikaans, Arabic, Brazilian Portuguese, Catalan, Chinese, Chinese Traditional, Croatian, Czech, Danish, Dutch, English, Finnish, French, German, Greek, Hungarian, Indonesian, Italian, Japanese (+En), Korean (+En), Lithuanian, Macedonian, Norwegian, Persian, Polish, Portuguese, Romanian, Russian, Serbian, Slovak, Slovene, Spanish, Swedish, and Ukrainian..

The table of information related to the supported languages follows. It is sorted by language alphabetically. The {\bf Status} column was generated from sources and shows approximately the last version when the translator was updated.



 
\begin{tabular}{|l|l|l|l|}
  \hline 
  {\bf Language} & {\bf Maintainer} & {\bf Contact address} & {\bf Status} \\
  \hline

  \hline
  Afrikaans & Johan Prinsloo & {\tt\tiny johan@zippysnoek.com} & 1.4.6 \\
  \hline
  Arabic & Moaz Reyad & {\tt\tiny moazreyad@yahoo.com} & 1.4.6 \\
  \hline
  Brazilian Portuguese & Fabio "FJTC" Jun Takada Chino & {\tt\tiny jun-chino@uol.com.br} & up-to-date \\
  \hline
  Catalan & Maximiliano Pin & {\tt\tiny mcpin@emtesistemas.com} & 1.5.4 \\
  ~ & Albert Mora & {\tt\tiny amora@iua.upf.es} & ~ \\
  \hline
  Chinese & Li Daobing & {\tt\tiny lidaobing@gmail.com} & up-to-date \\
  ~ & Wei Liu & {\tt\tiny liuwei@asiainfo.com} & ~ \\
  \hline
  Chinese Traditional & Daniel YC Lin & {\tt\tiny dlin.tw@gmail.com} & up-to-date \\
  ~ & Gary Lee & {\tt\tiny garywlee@gmail.com} & ~ \\
  \hline
  Croatian & Boris Bralo & {\tt\tiny boris.bralo@zg.htnet.hr} & up-to-date \\
  \hline
  Czech & Petr P\v{r}ikryl & {\tt\tiny prikrylp@skil.cz} & up-to-date \\
  \hline
  Danish & Erik S\o{}e S\o{}rensen & {\tt\tiny eriksoe+doxygen@daimi.au.dk} & 1.5.4 \\
  \hline
  Dutch & Dimitri van Heesch & {\tt\tiny dimitri@stack.nl} & up-to-date \\
  \hline
  English & Dimitri van Heesch & {\tt\tiny dimitri@stack.nl} & up-to-date \\
  \hline
  Finnish & Olli Korhonen & {\tt\tiny olli.korhonen lost@cyberspace} & obsolete \\
  \hline
  French & Xavier Outhier & {\tt\tiny xouthier@yahoo.fr} & 1.5.4 \\
  \hline
  German & Jens Seidel & {\tt\tiny jensseidel@users.sf.net} & up-to-date \\
  \hline
  Greek & Paul Gessos & {\tt\tiny nickreserved@yahoo.com} & 1.5.4 \\
  \hline
  Hungarian & \'{A}kos Kiss & {\tt\tiny akiss@users.sourceforge.net} & 1.4.6 \\
  ~ & F\"{o}ldv\'{a}ri Gy\"{o}rgy & {\tt\tiny foldvari lost@cyberspace} & ~ \\
  \hline
  Indonesian & Hendy Irawan & {\tt\tiny ceefour@gauldong.net} & 1.4.6 \\
  \hline
  Italian & Alessandro Falappa & {\tt\tiny alessandro@falappa.net} & up-to-date \\
  ~ & Ahmed Aldo Faisal & {\tt\tiny aaf23@cam.ac.uk} & ~ \\
  \hline
  Japanese & Ryunosuke Satoh & {\tt\tiny sun594@hotmail.com} & 1.5.4 \\
  ~ & Kenji Nagamatsu & {\tt\tiny naga@joyful.club.ne.jp} & ~ \\
  ~ & Iwasa Kazmi & {\tt\tiny iwasa@cosmo-system.jp} & ~ \\
  \hline
  JapaneseEn & see the Japanese language & {\tt\tiny ~} & English based \\
  \hline
  Korean & Kim Taedong & {\tt\tiny fly1004@gmail.com} & up-to-date \\
  ~ & SooYoung Jung & {\tt\tiny jung5000@gmail.com} & ~ \\
  ~ & Richard Kim & {\tt\tiny ryk@dspwiz.com} & ~ \\
  \hline
  KoreanEn & see the Korean language & {\tt\tiny ~} & English based \\
  \hline
  Lithuanian & Tomas Simonaitis & {\tt\tiny haden@homelan.lt} & 1.4.6 \\
  ~ & Mindaugas Radzius & {\tt\tiny mindaugasradzius@takas.lt} & ~ \\
  ~ & Aidas Berukstis & {\tt\tiny aidasber@takas.lt} & ~ \\
  \hline
  Macedonian & Slave Jovanovski & {\tt\tiny slavejovanovski@yahoo.com} & 1.5.04 \\
  \hline
  Norwegian & Lars Erik Jordet & {\tt\tiny lejordet@gmail.com} & 1.4.6 \\
  \hline
  Persian & Ali Nadalizadeh & {\tt\tiny nadalizadeh@gmail.com} & up-to-date \\
  \hline
  Polish & Piotr Kaminski & {\tt\tiny Piotr.Kaminski@ctm.gdynia.pl} & 1.4.6 \\
  ~ & Grzegorz Kowal & {\tt\tiny g\_kowal@poczta.onet.pl} & ~ \\
  \hline
  Portuguese & Rui Godinho Lopes & {\tt\tiny ruiglopes@yahoo.com} & 1.3.3 \\
  \hline
  Romanian & Alexandru Iosup & {\tt\tiny aiosup@yahoo.com} & 1.4.1 \\
  \hline
  Russian & Alexandr Chelpanov & {\tt\tiny cav@cryptopro.ru} & 1.5.4 \\
  \hline
  Serbian & Dejan Milosavljevic & {\tt\tiny dmilos@email.com} & 1.4.1 \\
  \hline
  Slovak & Stanislav Kudl\'{a}\v{c} & {\tt\tiny skudlac@pobox.sk} & 1.2.18 \\
  \hline
  Slovene & Matja\v{z} Ostrover\v{s}nik & {\tt\tiny matjaz.ostroversnik@ostri.org} & 1.4.6 \\
  \hline
  Spanish & Bartomeu & {\tt\tiny bartomeu@loteria3cornella.com} & up-to-date \\
  ~ & Francisco Oltra Thennet & {\tt\tiny foltra@puc.cl} & ~ \\
  \hline
  Swedish & Mikael Hallin & {\tt\tiny mikaelhallin@yahoo.se} & 1.4.6 \\
  \hline
  Ukrainian & Olexij Tkatchenko & {\tt\tiny olexij.tkatchenko@parcs.de} & 1.4.1 \\
  \hline
\end{tabular}


Most people on the list have indicated that they were also busy doing other things, so if you want to help to speed things up please let them (or me) know.

If you want to add support for a language that is not yet listed please read the next section.

\subsubsection*{Adding a new language to doxygen}

This short HOWTO explains how to add support for the new language to Doxygen:

Just follow these steps: \begin{enumerate}
\item Tell me for which language you want to add support. If no one else is already working on support for that language, you will be assigned as the maintainer for the language. \item Create a copy of translator\_\-en.h and name it translator\_\-$<$your\_\-2\_\-letter\_\-country\_\-code$>$.h I'll use xx in the rest of this document. \item Add definition of the symbol for your language into lang\_\-cfg.h: 

\footnotesize\begin{verbatim}
#define LANG_xx
\end{verbatim}
\normalsize
 Use capital letters for your {\tt xx} (to be consistent). The {\tt lang\_\-cfg.h} defines which language translators will be compiled into doxygen executable. It is a kind of configuration file. If you are sure that you do not need some of the languages, you can remove (comment out) definitions of symbols for the languages, or you can say {\tt \#undef} instead of {\tt \#define} for them. \item Edit language.cpp: Add a 

\footnotesize\begin{verbatim}
#ifdef LANG_xx
#include<translator_xx.h>
#endif
\end{verbatim}
\normalsize
 Remember to use the same symbol LANG\_\-xx that you added to {\tt lang\_\-cfg.h}. I.e., the {\tt xx} should be capital letters that identify your language. On the other hand, the {\tt xx} inside your {\tt translator\_\-xx.h} should use lower case. 

Now, in {\tt setTranslator()} add 

\footnotesize\begin{verbatim}
#ifdef LANG_xx
    else if (L_EQUAL("your_language_name"))
    {
      theTranslator = new TranslatorYourLanguage;
    }
#endif    
\end{verbatim}
\normalsize
 after the {\tt if \{ ... \}}. I.e., it must be placed after the code for creating the English translator at the beginning, and before the {\tt else \{ ... \}} part that creates the translator for the default language (English again). \item Edit libdoxygen.pro.in and add {\tt translator\_\-xx.h} to the {\tt HEADERS} line. \item Edit {\tt translator\_\-xx.h}: \begin{itemize}
\item Rename {\tt TRANSLATOR\_\-EN\_\-H} to {\tt TRANSLATOR\_\-XX\_\-H} twice (i.e. in the {\tt \#ifndef} and {\tt \#define} preprocessor commands at the beginning of the file). \item Rename TranslatorEnglish to TranslatorYourLanguage \item In the member {\tt idLanguage()} change \char`\"{}english\char`\"{} into the name of your language (use lower case characters only). Depending on the language you may also wish to change the member functions latexLanguageSupportCommand(), idLanguageCharset() and others (you will recognize them when you start the work). \item Edit all the strings that are returned by the member functions that start with tr. Try to match punctuation and capitals! To enter special characters (with accents) you can: \begin{itemize}
\item Enter them directly if your keyboard supports that and you are using a Latin-1 font. Doxygen will translate the characters to proper $\mbox{\LaTeX}$ and leave the HTML and man output for what it is (which is fine, if idLanguageCharset() is set correctly). \item Use html codes like \&auml; for an a with an umlaut (i.e. \"{a}). See the HTML specification for the codes. \end{itemize}
\end{itemize}
\item Run configure and make again from the root of the distribution, in order to regenerated the Makefiles. \item Now you can use {\tt OUTPUT\_\-LANGUAGE = your\_\-language\_\-name} in the config file to generate output in your language. \item Send {\tt translator\_\-xx.h} to me so I can add it to doxygen. Send also your name and e-mail address to be included in the {\tt maintainers.txt} list. \end{enumerate}


\subsubsection*{Maintaining a language}

New versions of doxygen may use new translated sentences. In such situation, the {\tt Translator} class requires implementation of new methods -- its interface changes. Of course, the English sentences need to be translated to the other languages. At least, new methods have to be implemented by the language-related translator class; otherwise, doxygen wouldn't even compile. Waiting until all language maintainers have translated the new sentences and sent the results would not be very practical. The following text describes the usage of translator adapters to solve the problem.

{\bf The role of Translator Adapters.} Whenever the {\tt Translator} class interface changes in the new release, the new class {\tt TranslatorAdapter\_\-x\_\-y\_\-z} is added to the {\tt translator\_\-adapter.h} file (here x, y, and z are numbers that correspond to the current official version of doxygen). All translators that previously derived from the {\tt Translator} class now derive from this adapter class.

The {\tt TranslatorAdapter\_\-x\_\-y\_\-z} class implements the new, required methods. If the new method replaces some similar but obsolete method(s) (e.g. if the number of arguments changed and/or the functionality of the older method was changed or enriched), the {\tt TranslatorAdapter\_\-x\_\-y\_\-z} class may use the obsolete method to get the result which is as close as possible to the older result in the target language. If it is not possible, the result (the default translation) is obtained using the English translator, which is (by definition) always up-to-date.

{\bf For example,} when the new {\tt trFile()} method with parameters (to determine the capitalization of the first letter and the singular/plural form) was introduced to replace the older method {\tt trFiles()} without arguments, the following code appeared in one of the translator adapter classes:



\footnotesize\begin{verbatim}
    /*! This is the default implementation of the obsolete method
     * used in the documentation of a group before the list of
     * links to documented files.  This is possibly localized.
     */
    virtual QCString trFiles()
    { return "Files"; }

    /*! This is the localized implementation of newer equivalent
     * using the obsolete method trFiles().
     */
    virtual QCString trFile(bool first_capital, bool singular)
    {
      if (first_capital && !singular)
        return trFiles();  // possibly localized, obsolete method
      else
        return english.trFile(first_capital, singular);
    }
\end{verbatim}
\normalsize


The {\tt trFiles()} is not present in the {\tt TranslatorEnglish} class, because it was removed as obsolete. However, it was used until now and its call was replaced by



\footnotesize\begin{verbatim}
    trFile(true, false)
\end{verbatim}
\normalsize


in the doxygen source files. Probably, many language translators implemented the obsolete method, so it perfectly makes sense to use the same language dependent result in those cases. The {\tt TranslatorEnglish} does not implement the old method. It derives from the abstract {\tt Translator} class. On the other hand, the old translator for a different language does not implement the new {\tt trFile()} method. Because of that it is derived from another base class -- {\tt TranslatorAdapter\_\-x\_\-y\_\-z}. The {\tt TranslatorAdapter\_\-x\_\-y\_\-z} class have to implement the new, required {\tt trFile()} method. However, the translator adapter would not be compiled if the {\tt trFiles()} method was not implemented. This is the reason for implementing the old method in the translator adapter class (using the same code, that was removed from the TranslatorEnglish).

The simplest way would be to pass the arguments to the English translator and to return its result. Instead, the adapter uses the old {\tt trFiles()} in one special case -- when the new {\tt trFile(true,~false)} is called. This is the mostly used case at the time of introducing the new method -- see above. While this may look too complicated, the technique allows the developers of the core sources to change the Translator interface, while the users may not even notice the change. Of course, when the new {\tt trFile()} is used with different arguments, the English result is returned and it will be noticed by non English users. Here the maintainer of the language translator should implement at least that one particular method.

{\bf What says the base class of a language translator?} If the language translator class inherits from any adapter class the maintenance is needed. In such case, the language translator is not considered up-to-date. On the other hand, if the language translator derives directly from the abstract class {\tt Translator}, the language translator is up-to-date.

The translator adapter classes are chained so that the older translator adapter class uses the one-step-newer translator adapter as the base class. The newer adapter does less {\em adapting\/} work than the older one. The oldest adapter class derives (indirectly) from all of the adapter classes. The name of the adapter class is chosen so that its suffix is derived from the previous official version of doxygen that did not need the adapter. This way, one can say approximately, when the language translator class was last updated -- see details below.

The newest translator adapter derives from the abstract {\tt TranslatorAdapterBase} class that derives directly from the abstract {\tt Translator} class. It adds only the private English-translator member for easy implementation of the default translation inside the adapter classes, and it also enforces implementation of one method for noticing the user that the language translation is not up-to-date (because of that some sentences in the generated files may appear in English).

Once the oldest adapter class is not used by any of the language translators, it can be removed from the doxygen project. The maintainers should try to reach the state with the minimal number of translator adapter classes.

{\bf To simplify the maintenance of the language translator classes} for the supported languages, the {\tt translator.py} Python script was developed (located in {\tt doxygen/doc} directory). It extracts the important information about obsolete and new methods from the source files for each of the languages. The information is stored in the {\em translator report\/} ASCII file (translator\_\-report.txt). 

Looking at the base class of the language translator, the script guesses also the status of the translator -- see the last column of the table with languages above. The {\tt translator.py} is called automatically when the doxygen documentation is generated. You can also run the script manualy whenever you feel that it can help you. Of course, you are not forced to use the results of the script. You can find the same information by looking at the adapter class and its base classes.

{\bf How should I update my language translator?} Firstly, you should be the language maintainer, or you should let him/her know about the changes. The following text was written for the language maintainers as the primary audience.

There are several approaches to be taken when updating your language. If you are not extremely busy, you should always chose the most radical one. When the update takes much more time than you expected, you can always decide use some suitable translator adapter to finish the changes later and still make your translator working.

{\bf The most radical way of updating the language translator} is to make your translator class derive directly from the abstract class {\tt Translator} and provide translations for the methods that are required to be implemented -- the compiler will tell you if you forgot to implement some of them. If you are in doubt, have a look at the {\tt TranslatorEnglish} class to recognize the purpose of the implemented method. Looking at the previously used adapter class may help you sometimes, but it can also be misleading because the adapter classes do implement also the obsolete methods (see the previous {\tt trFiles()} example).

In other words, the up-to-date language translators do not need the {\tt TranslatorAdapter\_\-x\_\-y\_\-z} classes at all, and you do not need to implement anything else than the methods required by the Translator class (i.e. the pure virtual methods of the {\tt Translator} -- they end with {\tt =0;}).

If everything compiles fine, try to run {\tt translator.py}, and have a look at the translator report (ASCII file) at the {\tt doxygen/doc} directory. Even if your translator is marked as up-to-date, there still may be some remarks related to your souce code. Namely, the obsolete methods--that are not used at all--may be listed in the section for your language. Simply, remove their code (and run the {\tt translator.py} again). Also, you will be informed when you forgot to change the base class of your translator class to some newer adapter class or directly to the Translator class.

{\bf If you do not have time to finish all the updates} you should still start with {\em the most radical approach\/} as described above. You can always change the base class to the translator adapter class that implements all of the not-yet-implemented methods.

{\bf If you prefer to update your translator gradually}, have a look at {\tt TranslatorEnglish} (the {\tt translator\_\-en.h} file). Inside, you will find the comments like {\tt new since 1.2.4} that separate always a number of methods that were implemented in the stated version. Do implement the group of methods that are placed below the comment that uses the same version numbers as your translator adapter class. (For example, your translator class have to use the {\tt TranslatorAdapter\_\-1\_\-2\_\-4}, if it does not implement the methods below the comment {\tt new since 1.2.4}. When you implement them, your class should use newer translator adapter.

Run the {\tt translator.py} script occasionaly and give it your {\tt xx} identification (from {\tt translator\_\-xx.h}) to create the translator report shorter (also produced faster) -- it will contain only the information related to your translator. Once you reach the state when the base class should be changed to some newer adapter, you will see the note in the translator report.

Warning: Don't forget to compile Doxygen to discover, whether it is compilable. The {\tt translator.py} does not check if everything is correct with respect to the compiler. Because of that, it may lie sometimes about the necessary base class.

{\bf The most obsolete language translators} would lead to implementation of too complicated adapters. Because of that, doxygen developers may decide to derive such translators from the {\tt TranslatorEnglish} class, which is by definition always up-to-date.

When doing so, all the missing methods will be replaced by the English translation. This means that not-implemented methods will always return the English result. Such translators are marked using word {\tt obsolete}. You should read it {\bf really obsolete}. No guess about the last update can be done.

Often, it is possible to construct better result from the obsolete methods. Because of that, the translator adapter classes should be used if possible. On the other hand, implementation of adapters for really obsolete translators brings too much maintenance and run-time overhead. 