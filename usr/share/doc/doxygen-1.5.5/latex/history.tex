\subsection*{Version 1.2.0}

\subsubsection*{Major new features:}

\begin{itemize}
\item Support for RTF output. \item Using the dot tool of the AT\&T's GraphViz package, doxygen can now generate inheritance diagrams, collaboration diagrams, include dependency graphs, included by graphs and graphical inheritance overviews. \item Function arguments can now be documented with separate comment blocks. \item Initializers and macro definitions are now included in the documentation. \item Variables and typedefs are now put in their own section. \item Old configuration files can be upgraded using the -u option without loosing any changes. \item Using the $\backslash$if and $\backslash$endif commands, doxygen can conditionally include documentation blocks. \item Added Doc++ like support for member grouping. \item Doxygen now has a GUI front-end called doxywizard (based on Qt-2.1) \item All info about configuration options is now concentrated in a new tool called configgen. This tool can generate the configuration parser and GUI front-end from source templates. \item Better support for the using keyword. \item New transparent mini logo that is put in the footer of all HTML pages. \item Internationalization support for the Polish, Portuguese and Croatian language. \item Todo list support. \item If the source browser is enabled, for a function, a list of function whose implementation calls that function, is generated. \item All source code fragments are now syntax highlighted in the HTML output. The colors can be changed using cascading style sheets. \end{itemize}


\subsection*{Version 1.0.0}

\subsubsection*{Major new features:}

\begin{itemize}
\item Support for templates and namespaces. \item Internationalization support. Currently supported languages are: English, Czech, German, Spanish, Finnish, French, Italian, Japanese, Dutch, and Swedish. \item Automatic generation of inheritance diagrams for sub and super classes. \item Support for man page, compressed HTML help, and hyperlinked PDF output. \item Cross-referencing documentation with source code and source inlining. \item LaTeX formulas can be included in the documentation. \item Support for parsing Corba and Microsoft IDL. \item Images can be included in the documentation. \item Improved parsing and preprocessing. \end{itemize}


\subsection*{Version 0.4}

\subsubsection*{Major new features:}

\begin{itemize}
\item LaTeX output generation. \item Full JavaDoc support. \item Build-in C-preprocessor for correct conditional parsing of source code that is read by Doxygen. \item Build-in HTML to LaTeX converter. This allows you to use HTML tags in your documentation, while doxygen still generates proper LaTeX output. \item Many new commands (there are now more than 60!) to document more entities, to make the documentation look nicer, and to include examples or pieces of examples. \item Enum types, enum values, typedefs, \#defines, and files can now be documented. \item Completely new documentation, that is now generated by Doxygen. \item A lot of small examples are now included. \end{itemize}


\subsection*{Version 0.3}

\subsubsection*{Major new features:}

\begin{itemize}
\item A PHP based search engine that allows you to search through the generated documentation. \item A configuration file instead of command-line options. A default configuration file can be generated by \href{doxygen_usage.html}{\tt doxygen}. \item Added an option to generate output for undocumented classes. \item Added an option to generate output for private members. \item Every page now contains a condensed index page, allowing much faster navigation through the documentation. \item Global and member variables can now be documented. \item A project name can now given, which will be included in the documentation. \end{itemize}


\subsection*{Version 0.2}

\subsubsection*{Major new features:}

\begin{itemize}
\item Blocks of code are now parsed. Function calls and variables are replaced by links to their documentation if possible. \item Special example documentation block added. This can be used to provide cross references between the documentation and some example code. \item Documentation blocks can now be placed inside the body of a class. \item Documentation blocks with line range may now be created using special {\tt //!} C++ line comments. \item Unrelated members can now be documented. A page containing a list of these members is generated. \item Added an {\tt $\backslash$include} command to insert blocks of source code into the documentation. \item Warnings are generated for members that are undocumented. \item You can now specify your own HTML headers and footers for the generated pages. \item Option added to generated indices containing all external classes instead of only the used ones. \end{itemize}


\subsection*{Version 0.1}

Initial version. 